%!TEX program=xelatex
\documentclass[UTF8]{ctexart}
\usepackage{amsmath,mathrsfs,amsfonts,amsbsy,amsthm}
\usepackage{xcolor,tcolorbox}
\tcbuselibrary{most}

\usepackage{enumerate,graphicx,wrapfig}
\usepackage{cases,subeqnarray}
\usepackage{hyperref}
\usepackage[left=1.2in,right=1.2in,top=1.8in,bottom=1.8in]{geometry}

\newcommand{\e}{\mathrm{e}}
\renewcommand{\d}{\mathrm{d}}
\newcommand{\p}[2]{\frac{\partial #1}{\partial #2}}
\newcommand{\x}{\boldsymbol{x}}
\newcommand{\y}{\boldsymbol{y}}


\title{数学分析笔记}
\author{管思桐}
\date{}

\begin{document}
\maketitle
\tableofcontents
\newpage

\section{多元函数的微分学}

       \subsection{隐函数定理}
        \paragraph{\colorbox{lime}{定义}}若$D\subset \mathrm{R}^2$是一个区域,$F(x,y)$是$D$上的一个二元函数,而且$F(x_0,y_0)=0,(x_0,y_0)\in D$,如果在$(x_0,y_0)$附近,由方程
        $$F(x,y)=0$$
        可以唯一确定一个函数$y=f(x)$使得$f(x_0)=y_0$,$F(x,f(x))=0,x\in(x_0-\delta,x_0+\delta)$,则称$f(x)$是由$F(x,y)=0$确定的\textbf{隐函数}。
        \paragraph{\colorbox{pink}{例}}$y^5+7y-x^3=0$
        \paragraph{\colorbox{pink}{例}}Kepler方程:$$y-\varepsilon\sin y-x=0$$
        \paragraph{\colorbox{pink}{例}}$F(x,y)=x^2+y^2-1$
        
        \paragraph{\colorbox{lime}{定理}}设$F(x,y)$满足以下条件:
            \begin{enumerate}[(1)]
                \item  $F(x_0,y_0)=0$;
                \item 在$D=\{(x,y):|x-x_0|\le a,|y-y_0|\le b\}$上,$F\in C(D)$且$\partial_xF,\partial_yF\in C(D)$;
                \item $\partial_yF(x_0,y_0)\not =0$,
            \end{enumerate}
        则$\exists\rho>0,\eta>0$使得
        \begin{enumerate}[(1)]
            \item $x\in(x_0-\rho,x_0+\rho)$,方程$F(x,y)=0$在$(y_0-\eta,y_0+\eta)$中有唯一解$y=f(x)$;
            \item $f(x_0)=y_0$;
            \item $f(x)\in C((x_0-\rho,x_0+\rho))$;
            \item $f(x)$在$(x_0-\rho,x_0+\rho)$上连续可导,而且$$f'(x)=-\frac{\partial_xF(x,f(x))}{\partial_yF(x,f(x))},x\in (x_0-\rho,x_0+\rho).$$
        \end{enumerate}
        \begin{proof}
            利用
            \begin{enumerate}
                \item 导数$>0\Rightarrow$函数严格单调增加;
                \item 连续函数的介值定理
            \end{enumerate}
            \paragraph{Step 1}存在性:不妨设$\partial_yF(x_0,y_0)>0$,则由于$\partial_yF\in C(D)$,故$\exists 0<\alpha\le a,0<\beta\le b$使得$\partial_yF(x_0,y_0)$在$D^*=\{(x,y):|x-x_0|\le \alpha,|y-y_0|\le \beta\}$上$>0$。由于$\partial_yF(x_0,y)>0,|y-y_0|\le\beta$,故$F(x_0,y)$在$[y_0-\beta,y_0+\beta]$上严格单调增加,又$F(x_0,y_0)=0$,故$F(x_0,y_0-\beta)<0,F(x_0,y_0+\beta)>0$.又$F(x,y_0-\beta),F(x,y_0+\beta)$在$[x_0-\alpha,x_0+\alpha]$上连续,故$\exists\rho\in(0,\alpha)$使得
            $$F(x,y_0+\beta)>0,F(x,y_0-\beta)<0,x\in[x_0-\rho,x_0+\rho]$$
            又$\forall\bar{x}\in(x_0-\rho,x_0+\rho),\partial_yF(\bar{x},y)>0,y\in[y_0-\beta,y_0+\beta],F(\bar{x},y)$关于$y$在$[y_0-\beta,y_0+\beta]$上严格单调增加,故$\exists!\bar{y}\in(y_0-\beta,y_0+\beta)$试得$F(\bar{x},\bar{y})=0$.记$\bar{y}=f(\bar{x})$

            \paragraph{Step 2}$f(x)$在$(x_0-\rho,x_0+\rho)$上连续:任取$\bar{x}\in(x_0-\rho,x_0+\rho)$,我们根据定义证明$f(x)$在$\bar{x}$处连续,即$\forall\varepsilon>0,\exists\delta>0,$只要$|x-\bar{x}|<\delta$且$x\in(x_0-\rho,x_0+\rho)$,就有$|f(x)-f(\bar{x})|<\varepsilon$.

            由于$F(\bar{x},f(\bar{x}))=0$,而且$F(\bar{x},y)$关于$y$严格单调增加,所以$$F(\bar{x},f(\bar{x})+\varepsilon)>0,F(\bar{x},f(\bar{x})-\varepsilon)<0$$
            又$F(x,f(\bar{x})+\varepsilon),F(x,f(\bar{x})-\varepsilon)$在$(x_0-\rho,x_0+\rho)$上连续,故$\exists\delta=\delta(\varepsilon)>0,s.t.$
            $$F(x,f(\bar{x})+\varepsilon)>0,F(x,f(\bar{x})-\varepsilon)<0,|x-\bar{x}|<\delta,x\in(x_0-\rho,x_0+\rho)$$
            由$F(x,y)$关于$y$严格单调增加,故
            $$f(\bar{x})-\varepsilon<f(x)<f(\bar{x})+\varepsilon\Rightarrow|f(x)-f(\bar{x})|<\varepsilon$$
            (\textbf{Note:}$f$连续性不需要$F(x,y)$关于$x$可偏导这一条件)

            \paragraph{连续性的另一个证明}任取$\bar{x}\in(x_0-\rho,x_0+\rho)$,取$\Delta x$足$0<|\Delta x|<<1,s.t.\bar{x}+\Delta x\in(x_0-\rho,x_0+\rho)$.
            由于$$F(\bar{x},f(\bar{x}))=0,F(\bar{x}+\Delta x,f(\bar{x}+\Delta x))=0,$$
            故
            \begin{align*}
                0=&F(\bar{x}+\Delta x,f(\bar{x}+\Delta x))-F(\bar{x},f(\bar{x}))\\
                =&F(\bar{x}+\Delta x,f(\bar{x}+\Delta x))-F(\bar{x}+\Delta x,f(\bar{x}))+F(\bar{x}+\Delta x,f(\bar{x}))-F(\bar{x},f(\bar{x}))\\
                =&\partial_yF(\bar{x}+\Delta x,\theta f(\bar{x}+\Delta x)+(1-\theta)f(\bar{x}))[f(\bar{x}+\Delta x)-f(\bar{x})]+F(\bar{x}+\Delta x,f(\bar{x}))-F(\bar{x},f(\bar{x}))
            \end{align*}
            由于$F(\bar{x}+\Delta x,f(\bar{x}))-F(\bar{x},f(\bar{x}))\in D^*$,故$\not=0$,从而
            $$f(\bar{x}+\Delta x)-f(\bar{x})=-\frac{F(\bar{x}+\Delta x,f(\bar{x}))-F(\bar{x},f(\bar{x}))}{\partial_yF(\bar{x}+\Delta x,\theta f(\bar{x}+\Delta x)+(1-\theta)f(\bar{x}))}$$
            故
            $$|f(\bar{x}+\Delta x)-f(\bar{x})|\le\frac{|F(\bar{x}+\Delta x,f(\bar{x}))-F(\bar{x},f(\bar{x}))|}{m}$$
            其中$m=\inf_{x\in D^*}\partial_yF>0$.所以$\lim_{\Delta x\to 0}f(\bar{x}+\Delta x)-f(\bar{x})=0$.



            \paragraph{Step 3}$f$在$(x_0-\rho,x_0+\rho)$上可偏导:任取$\bar{x}\in(x_0-\rho,x_0+\rho)$,取$\Delta x$满足$0<|\Delta x|<<1,s.t.\bar{x}+\Delta x\in(x_0-\rho,x_0+\rho)$.由于$$F(\bar{x},f(\bar{x}))=0,F(\bar{x}+\Delta x,f(\bar{x}+\Delta x))=0,$$
            同理有
            \begin{align*}
                0=&F(\bar{x}+\Delta x,f(\bar{x}+\Delta x))-F(\bar{x},f(\bar{x}))\\
                =&\partial_xF(\bar{x}+\theta\Delta x,\theta f(\bar{x}+\Delta x)+(1-\theta)f(\bar{x}))\Delta x\\
                &+\partial_yF(\bar{x}+\theta\Delta x,\theta f(\bar{x}+\Delta x)+(1-\theta)f(\bar{x}))[f(\bar{x}+\Delta x)-f(\bar{x})]
            \end{align*}
            从而
            $$\frac{f(\bar{x}+\Delta x)-f(\bar{x})}{\Delta x}=-\frac{\partial_xF(\bar{x}+\theta\Delta x,\theta f(\bar{x}+\Delta x)+(1-\theta)f(\bar{x}))}{\partial_yF(\bar{x}+\theta\Delta x,\theta f(\bar{x}+\Delta x)+(1-\theta)f(\bar{x}))}$$
            即
            $$f'(\bar{x})=-\frac{\partial_xF(\bar{x},f(\bar{x}))}{\partial_yF(\bar{x},f(\bar{x}))}.$$
        \end{proof}

        \paragraph{\colorbox{orange!70}{Note:}}
        \begin{enumerate}[(1)]
            \item 隐函数定理是一个局部性定理,即只在$(x_0,y_0)$的一个邻域内成立;
            \item $\partial_yF(x_0,y_0)\not=0$只是充分条件。例:$F(x,y)=y^3-x=0$在$(0,0)$附近唯一确定隐函数,但$\partial_yF(0,0)=0$;
            \item 定理中的$x$与$y$的地位是平等的,即如果$\partial_yF(x_0,y_0)\not=0$,则在$(x_0,y_0)$的一个邻域中,$F(x,y)=0$可以唯一确定一个隐函数$x=g(y)$($F(g(y),y)=0$);
            \item 只是存在唯一性,可微性,但一般情况下很难写出$f(x)$的显示表达式;
            \item \textbf{\colorbox{lime}{推论}}高阶可微性$(C^k)$:
            
            若$F\in C^k(D)$,则$f(x)\in C^k((x_0-\rho,x_0+\rho))$,$k=1,2,\cdots.$若$F\in C^\omega(D)$,\footnote{$f\in C^\omega(D)$:$f$为解析函数}则$f(x)\in C^\omega(D)$

        \begin{proof}
            利用归纳法证明$f(x)\in C^k((x_0-\rho,x_0+\rho))$:

            当$F\in C^1(D)$时,$f'(x)=-\frac{\partial_xF(x,f(x))}{\partial_yF(x,f(x))},x\in(x_0-\rho,x_0+\rho)$,此时$f\in C^1((x_0-\rho,x_0+\rho))$.

            当$F\in C^2(D)$时,$\partial_xF,\partial_yF\in C^1(D)$,由复合函数的可微性知$f'(x)\in C^1((x_0-\rho,x_0+\rho))$也即$f(x)\in C^2((x_0-\rho,x_0+\rho))$.

            假设$F\in C^k(D)\Rightarrow f(x)\in C^k((x_0-\rho,x_0+\rho))$。那么$F\in C^{k+1}(D)$时,有$\partial_xF,\partial_yF\in C^k(D)$,故$\frac{\partial_xF(x,f(x))}{\partial_yF(x,f(x))}\in C^k(x_0-\rho,x_0+\rho)$,即$f'(x)\in C^k(x_0-\rho,x_0+\rho)$,于是$f(x)\in C^{k+1}((x_0-\rho,x_0+\rho))$.

            $f(x)\in C^\omega(D)$的证明比较困难,这里从略。
        \end{proof}
            \item 若将$\partial_yF(x_0,y_0)\not=0$换成:$\forall\bar{x}\in[x_0-\alpha,x_0+\alpha],F(\bar{x},y)$关于$y$是严格单调增加的,则也$\exists!$连续隐函数(也即只假设$F\in C(D),F(x_0,y_0)=0$)。
        \end{enumerate}
        
    \paragraph{\colorbox{lime}{多元隐函数定理}}设$n+1$元函数$F(\boldsymbol{x},y)$($\boldsymbol{x}=(x_1,x_2,\cdots,x_n)$)满足:
    \begin{enumerate}[(1)]
        \item $F(\boldsymbol{x}_0,y_0)=0$;
        \item 记$D=\{(\boldsymbol{x},y):|x_i-x_i^0|\le a,|y-y_0|\le b,i=1,2,\cdots,n\},F\in C^1(D)$;
        \item $\partial_yF(\boldsymbol{x}_0,y_0)\not=0$
    \end{enumerate}
    则$\exists\rho>0,\eta>0$,使得
    \begin{enumerate}[(i)]
        \item $\forall\boldsymbol{x}_0\in O(\bar{x}_0,\rho)$,方程$F(\boldsymbol{x}_0,y_0)=0$在$(y_0-\eta,y_0+\eta)$中存在唯一解$f(\boldsymbol{x})$;
        \item $f(\boldsymbol{x}_0)=y_0$;
        \item $f\in C(O(\boldsymbol{x}_0,\rho))$;
        \item $f\in C^1(O(\boldsymbol{x}_0,\rho))$且$$\frac{\partial f}{\partial x_i}=-\frac{\partial_{x_i}F(\boldsymbol{x},f(\boldsymbol{x}))}{\partial_yF(\boldsymbol{x},f(\boldsymbol{x}))}.$$
    \end{enumerate}
    \vspace{1ex}
    (iv)的证明:由$F(\boldsymbol{x},f(\boldsymbol{x}))=0$知:
    \begin{align*}
        &\partial_{x_i}F(\boldsymbol{x},f(\boldsymbol{x}))+\frac{\partial F}{\partial y}(\boldsymbol{x},f(\boldsymbol{x}))\frac{\partial f}{\partial x_i}(\boldsymbol{x})=0\\
        \Rightarrow &\frac{\partial f}{\partial x_i}=-\frac{\partial_{x_i}F(\boldsymbol{x},f(\boldsymbol{x}))}{\partial_yF(\boldsymbol{x},f(\boldsymbol{x}))}.
    \end{align*}

    \paragraph{\colorbox{cyan!70}{求导方法}}$F(x,f(x))=0,x\in(x_0-\rho,x_0+\rho),f(x_0)=y_0$,关于$x$求导得:
    \begin{align}
        &\frac{\partial F}{\partial x}(x,f(x))+\frac{\partial F}{\partial y}(x,f(x))\frac{\partial f}{\partial x}(x)=0\label{1}\\
        &0\Rightarrow f'(x)=-\frac{\partial_{x}F(x,f(x))}{\partial_yF(x,f(x))}\label{2}
    \end{align}
    从而可得$f'(x_0)$的值。

    二阶导可以直接由式\eqref{2}$f'(x)$出发求导;或可根据式\eqref{1}:
    $$0=\frac{\partial^2 F}{\partial x^2}(x,f(x))+2\frac{\partial^2F}{\partial x\partial y}(x,f(x))f'(x)+\frac{\partial^2F}{\partial y^2}(x,f(x))(f'(x))^2+\frac{\partial F}{\partial y}(x,f(x))f''(x)$$
    从而得到$f''(x)$以及$f''(x_0)$.


    \paragraph{\colorbox{cyan!70}{多元隐函数的求导方法}}
        \begin{equation}
            F(x_1,x_2,\cdots,x_n,f(x_1,x_2,\cdots,x_n))=0,x\in O(\boldsymbol{x}_0,\rho) \label{eq1}
        \end{equation}
        对\eqref{eq1}关于$x_i$求偏导得:
        \begin{equation}
            \partial_{x_i}F(\boldsymbol{x},f(\boldsymbol{x}))+\frac{\partial F}{\partial y}(\boldsymbol{x},f(\boldsymbol{x}))\frac{\partial f}{\partial x_i}(\boldsymbol{x})=0 \label{eq2}
        \end{equation}
        从而
        \begin{equation}
            \frac{\partial f}{\partial x_i}=-\frac{\partial_{x_i}F(\boldsymbol{x},f(\boldsymbol{x}))}{\partial_yF(\boldsymbol{x},f(\boldsymbol{x}))},i=1,2,\cdots,n \label{eq3}
        \end{equation}
        若$F\in C^2$,则$f\in C^2( O(\boldsymbol{x}_0,\rho))$,对\eqref{eq2}关于$x_j$求偏导,有:
        \begin{align}
            &\frac{\partial^2F}{\partial x_i\partial x_j}(\boldsymbol{x},f(\boldsymbol{x}))+\frac{\partial^2F}{\partial x_i\partial y}(\boldsymbol{x},f(\boldsymbol{x}))\frac{\partial f}{\partial x_j}(\boldsymbol{x})+\frac{\partial^2F}{\partial x_j\partial y}(\boldsymbol{x},f(\boldsymbol{x}))\frac{\partial f}{\partial x_i}(\boldsymbol{x})\\
            +&\frac{\partial^2F}{\partial y^2}(\boldsymbol{x},f(\boldsymbol{x}))\frac{\partial f}{\partial x_j}(\boldsymbol{x})\frac{\partial f}{\partial x_i}(\boldsymbol{x})+\frac{\partial F}{\partial y}(\boldsymbol{x},f(\boldsymbol{x}))\frac{\partial^2f}{\partial x_i\partial x_j}(\boldsymbol{x})=0
        \end{align}

        \paragraph{\colorbox{pink}{例}}$\sin x+(1-x)\ln y-xy^3=0$

        解:$F(x,y)=\sin x+(1-x)\ln y-xy^3=0$,$F(0,1)=0,\partial_yF(0,1)=\left.\left(\frac{1-x}{y}-3xy^2\right)\right|_{(0,1)}=1$,$F(1,(\sin 1)^{1/5})=0,\partial_yF(1,(\sin 1)^{1/5})=\left.\left(\frac{1-x}{y}-3xy^2\right)\right|_{1,(\sin 1)^{1/5}}=-3(\sin 1)^{2/5}$.可以利用隐函数求导求出函数图像的部分点.
        


        \paragraph{\colorbox{pink}{例}}$x^2+y^2+z^2=4z$,确定$z$为$x,y$的函数,求$\frac{\partial^2z}{\partial x^2},\frac{\partial^2z}{\partial x\partial y}$(课本例题)

        解:关于$x,y$分别求偏导数,得:
        \begin{subequations}
            \begin{equation}
            2x+2z\frac{\partial z}{\partial x}=4\frac{\partial z}{\partial x} \label{eq4}
            \end{equation}
            \begin{equation}
            2y+2z\frac{\partial z}{\partial y}=4\frac{\partial z}{\partial y}\label{eq5}
            \end{equation}
        \end{subequations}
        从上式中可解出$\frac{\partial z}{\partial x}$和$\frac{\partial z}{\partial y}$.下面可以对\eqref{eq4}和\eqref{eq5}对$x$求偏导得到$\frac{\partial^2z}{\partial x^2},\frac{\partial^2z}{\partial x\partial y}$的值。



        \paragraph{\colorbox{orange!70}{几点补充}}
        \begin{enumerate}
            \item 唯一性的另一个证明(归一法):
            \begin{proof}
                假设$f_1(x),f_2(x)$是由$F(x,y)=0$在$(x_0,y_0)$的一个邻域中确定的两个隐函数,即$F(x,f_1(x))=F(x,f_2(x))=0;f_1(x_0)=f_2(x_0)=y_0$,$|f_1(x)-y_0|<\eta,|f_2(x)-y_0|<\eta,x\in O(x_0,\rho)$(下面利用中值定理证明:)
                $$\Rightarrow 0=F(x,f_1(x))-F(x,f_2(x))=\partial_yF(x,\theta f_1(x)+(1-\theta)f_2(x))[f_1(x)-f_2(x)],$$
                $\theta\in(0,1)$。由于$\partial_yF(x,\theta f_1(x)+(1-\theta)f_2(x))\not=0,x\in O(x_0,\rho)$,故$f_1(x)=f_2(x),x\in O(x_0,\rho)$
            \end{proof}
            \item 存在性的另一个证明(Picard迭代):
            \begin{enumerate}[{Step }1]
                \item 构造Picard序列;
                \item $\{y_n(x)\}$在$O(x_0,\rho)$上一致收敛。
            \end{enumerate}
        \end{enumerate}

        \paragraph{\colorbox{lime}{一元向量值函数的隐函数定理}}设$F(x,y_1,y_2),G(x,y_1,y_2)$满足以下条件:
        \begin{enumerate}[(1)]
            \item $F(x_0,y_1^0,y_2^0)=G(x_0,y_1^0,y_2^0)=0$;
            \item $D=\{(x,y_1,y_2):|x-x_0|<a,|y_1-y_1^0|<b_1,|y_2-y_2^0|<b_2\};F,G$在$D$上连续,而且有连续偏导数;
            \item Jacobi行列式不为零:$$\frac{\partial(F,G)}{\partial(y_1,y_2)}(x_0,y_1^0,y_2^0)=\begin{vmatrix}
                \frac{\partial F}{\partial y_1} & \frac{\partial F}{\partial y_2}\\
                \frac{\partial G}{\partial y_1} & \frac{\partial G}{\partial y_2}
            \end{vmatrix}(x_0,y_1^0,y_2^0)\not=0$$
        \end{enumerate}
        则$\exists\rho>0,\eta>0,s.t.$
        \begin{enumerate}[(i)]
            \item $\forall x\in(x_0-\rho,x_0+\rho)$,方程
            $$\begin{cases}
                F(x,y_1,y_2)=0\\
                G(x,y_1,y_2)=0
            \end{cases}$$
            在$O(\boldsymbol{y}_0,\eta)$中有唯一解(其中$\boldsymbol{y}_0=(y_1^0,y_2^0)$),记为$\boldsymbol{y}(x)=(y_1(x),y_2(x))$;
            \item $\boldsymbol{y}(x_0)=\boldsymbol{y}_0$;
            \item $\boldsymbol{y}(x)\in C^1(O(x_0,\rho))$(连续可微;$\boldsymbol{y}(x)\in C^1$意为$y_1(x),y_2(x)\in C^1$);
            \item 而且$$\begin{pmatrix}
                y'_1(x)\\
                y'_2(x)
            \end{pmatrix}=-\begin{pmatrix}
                \frac{\partial F}{\partial y_1} & \frac{\partial F}{\partial y_2}\\
                \frac{\partial G}{\partial y_1} & \frac{\partial G}{\partial y_2}
            \end{pmatrix}^{-1}(x,\boldsymbol{y}(x))\begin{pmatrix}
                \frac{\partial F}{\partial x}\\
                \frac{\partial G}{\partial x}
            \end{pmatrix}(x,\boldsymbol{y}(x)).$$
        \end{enumerate}
        
        \begin{proof}
            (思想:消元法)由于$$\begin{vmatrix}
                \frac{\partial F}{\partial y_1} & \frac{\partial F}{\partial y_2}\\
                \frac{\partial G}{\partial y_1} & \frac{\partial G}{\partial y_2}
            \end{vmatrix}(x_0,y_1^0,y_2^0)\not=0,$$
            故$(\frac{\partial G}{\partial y_1},\frac{\partial G}{\partial y_2})\not=(0,0)$;不妨设$\frac{\partial G}{\partial y_2}\not=0$.又$G(x_0,y_1,y_2)$在$(x,y_1^0,y_2^0)$的一个邻域,于是满足隐函数定理条件,即:
            \begin{enumerate}[(1)]
                \item $G(x_0,y_1^0,y_2^0)=0$;
                \item $G$在$D$上连续可微;
                \item $\frac{\partial G}{\partial y_2}(x_0,y_1^0,y_2^0)\not=0$
            \end{enumerate}
            由隐函数定理知$\exists\tilde{\rho}>0,\tilde{\eta}>0$以及隐函数$h(x,y_1)$满足:
            \begin{enumerate}[(i)]
                \item $h(x_0,y_1^0)=y_2^0$;
                \item $G(x,y_1,h(x,y_1))=0,(x,y_1)\in O((x_0,y_1^0),\tilde{\rho})$;
                \item $h(x,y_1)\in C^1(O((x_0,y_1^0),\tilde{\rho}))$;
                \item $\frac{\partial h}{\partial y_1}(x,y_1)=-\frac{\frac{\partial G}{\partial y_1}(x,y,h(x,y_1))}{\frac{\partial G}{\partial y_2}(x,y,h(x,y_1))},|h(x,y_1)-y_2^0|<\tilde{\eta},\forall (x,y_1)\in O((x_0,y_1^0),\tilde{\rho})$
            \end{enumerate}
            令$H(x,y_1)=F(x,y_1,h(x,y_1))$,则
            \begin{enumerate}[(i)]
                \item $H(x_0,y_1^0)=F(x_0,y_1^0,h(x_0,y_1^0))=F(x_0,y_1^0,y_2^0)=0$;
                \item 令$D=\{(x,y_1):|x-x_0|\le\frac{\tilde{\rho}}{2},|y-y_0|\le\frac{\tilde{\rho}}{2}\}$,则$H(x,y_1)$在$D$上连续可微(利用复合函数);
                \item $$\frac{\partial H}{\partial y_1}(x_0,y_1^0)=\frac{\partial F}{\partial y_1}(x_0,y_1^0,y_2^0)+\frac{\partial F}{\partial y_2}(x_0,y_1^0,y_2^0)\frac{\partial h}{\partial y_1}(x_0,y_1^0)=\frac{\frac{\partial(F,G)}{\partial(y_1,y_2)}(x_0,y_1^0,y_2^0)}{\frac{\partial G}{\partial y_2}(x_0,y_1^0,y_2^0)}\not=0$$
            \end{enumerate}
            对$H(x,y_1)$用隐函数定理,得:$\exists\rho>0,\eta>0$以及隐函数$y_1(x)\in C^1(O(x_0,\rho))$,满足:
            \begin{enumerate}[(i)]
                \item $y_1(x_0)=y_1^0,H(x,y_1(x))=0,|y_1(x)-y_1^0|<\eta,\forall x\in O(x_0,\rho)$;
                \item 连续可微;
                \item 可偏导(且有公式,不写了)。
            \end{enumerate}
            令$y_2(x)=h(x,y_1(x)),x\in O(x_0,\rho),$则$y_2(x_0)=h(x_0,y_1(x_0))=h(x_0,y_1^0)=y_2^0$,而且$y_2(x)\in C^1(O(x_0,\rho))$.

            又因为$F(x,y_1(x),y_2(x))=F(x,y_1(x),h(x,y_1(x)))=0,\forall x\in O(x_0,\rho)$,同理\\
            $G(x,y_1(x),y_2(x))=G(x,y_1(x),h(x,y_1(x)))=0,\forall x\in O(x_0,\rho)$;从而我们证明了定理中的存在性以及(ii)(iii).

            \textbf{唯一性的证明:}若$\boldsymbol{z}_1(x),\boldsymbol{z}_2(x)$是\footnote{发现这里$z$和$\boldsymbol{z}$差别不大。。但注意以下均为粗体的$\boldsymbol{z}$;另外下面行向量和列向量可能有点乱,可以稍微注意一下哈哈}由$$\begin{cases}
                F(x,\boldsymbol{y})=0\\
                G(x,\boldsymbol{y})=0
            \end{cases}$$
            在$(x_0,\boldsymbol{y}_0)$的一个邻域中有确定的两个隐函数,则$$\begin{cases}
                F(x,\boldsymbol{z}_1(x))=F(x,\boldsymbol{z}_2(x))=0\\
                G(x,\boldsymbol{z}_1(x))=G(x,\boldsymbol{z}_2(x))=0
            \end{cases}$$

            若直接用类似二元函数$f(x,y)$的证明的中值定理(见“几点补充”):
            \begin{subequations}
                \begin{equation}
                    0=F(x,\boldsymbol{z}_1(x))-F(x,\boldsymbol{z}_2(x))=\nabla_{\boldsymbol{y}}F(x,\theta\boldsymbol{z}_1(x)+(1-\theta)\boldsymbol{z}_2(x))(\boldsymbol{z}_1(x)-\boldsymbol{z}_2(x))
                \end{equation}
                \begin{equation}
                    0=G(x,\boldsymbol{z}_1(x))-G(x,\boldsymbol{z}_2(x))=\nabla_{\boldsymbol{y}}G(x,\tilde{\theta}\boldsymbol{z}_1(x)+(1-\tilde{\theta})\boldsymbol{z}_2(x))(\boldsymbol{z}_1(x)-\boldsymbol{z}_2(x))
                \end{equation}
            \end{subequations}
            由于$\theta$和$\tilde{\theta}$不一定相等,故不能用这种方法证明(我们想要利用“Jacobi矩阵行列式不为零”这一条件证明)。

            所以我们用Taylor公式:
            \begin{align}
                0=&F(x,\boldsymbol{z}_1(x))-F(x,\boldsymbol{z}_2(x))\\
                =&\nabla_{\boldsymbol{y}} F(x,\boldsymbol{z}_2(x))(\boldsymbol{z}_1(x)-\boldsymbol{z}_2(x))+o(||\boldsymbol{z}_1(x)-\boldsymbol{z}_2(x)||) \quad (x\to x_0)
            \end{align}
            \begin{align}
                0=&G(x,\boldsymbol{z}_1(x))-G(x,\boldsymbol{z}_2(x))\\
                =&\nabla_{\boldsymbol{y}} G(x,\boldsymbol{z}_2(x))(\boldsymbol{z}_1(x)-\boldsymbol{z}_2(x))+o(||\boldsymbol{z}_1(x)-\boldsymbol{z}_2(x)||) \quad (x\to x_0)
            \end{align}
            从而
            $$\boldsymbol{z}_1(x)-\boldsymbol{z}_2(x)=-\begin{pmatrix}
                \frac{\partial F}{\partial y_1} & \frac{\partial F}{\partial y_2}\\
                \frac{\partial G}{\partial y_1} & \frac{\partial G}{\partial y_2}
            \end{pmatrix}^{-1}\begin{pmatrix}
                o_1(||\boldsymbol{z}_1(x)-\boldsymbol{z}_2(x)||)\\
                o_2(||\boldsymbol{z}_1(x)-\boldsymbol{z}_2(x)||)
            \end{pmatrix},x\to x_0,$$
            两边取模,因$$\begin{vmatrix}
                \frac{\partial F}{\partial y_1} & \frac{\partial F}{\partial y_2}\\
                \frac{\partial G}{\partial y_1} & \frac{\partial G}{\partial y_2}
            \end{vmatrix}^{-1}$$为有界量,而
            $$\begin{vmatrix}
                o_1(||\boldsymbol{z}_1(x)-\boldsymbol{z}_2(x)||)\\
                o_2(||\boldsymbol{z}_1(x)-\boldsymbol{z}_2(x)||)
            \end{vmatrix}$$是关于$||\boldsymbol{z}_1(x)-\boldsymbol{z}_2(x)||$的高阶无穷小;\label{无穷小}故当$x\to x_0$时有
            $$\begin{vmatrix}
                \frac{\partial F}{\partial y_1} & \frac{\partial F}{\partial y_2}\\
                \frac{\partial G}{\partial y_1} & \frac{\partial G}{\partial y_2}
            \end{vmatrix}^{-1}\begin{vmatrix}
                o_1(||\boldsymbol{z}_1(x)-\boldsymbol{z}_2(x)||)\\
                o_2(||\boldsymbol{z}_1(x)-\boldsymbol{z}_2(x)||)
            \end{vmatrix}\le \frac{1}{2}||\boldsymbol{z}_1(x)-\boldsymbol{z}_2(x)||,$$
            从而$x\to x_0$时,有:
            $$||\boldsymbol{z}_1(x)-\boldsymbol{z}_2(x)||\le \frac{1}{2}||\boldsymbol{z}_1(x)-\boldsymbol{z}_2(x)||,$$
            从而只能有$||\boldsymbol{z}_1(x)-\boldsymbol{z}_2(x)||=0$,即$\boldsymbol{z}_1(x)=\boldsymbol{z}_2(x)$.
            
        \textbf{求导方法:}设$y_1(x),y_2(x)$是由方程组$$\begin{cases}
            F(x,y_1,y_2)=0\\
            G(x,y_1,y_2)=0
        \end{cases}$$在$(x_0,y_1^0,y_2^0)$的一个邻域内确定的隐函数,则有
        $$\begin{cases}
            F(x,y_1,y_2)=0\\
            G(x,y_1,y_2)=0
        \end{cases},\forall x\in O(x_0,\rho),$$
        对上式关于$x$求导得:
        $$\begin{cases}
            \frac{\partial F}{\partial x}(x,\boldsymbol{y}(x))+\frac{\partial F}{\partial y_1}(x,\boldsymbol{y}(x))y'_1(x)+\frac{\partial F}{\partial y_2}(x,\boldsymbol{y}(x))y'_2(x)=0\\
            \frac{\partial G}{\partial x}(x,\boldsymbol{y}(x))+\frac{\partial G}{\partial y_1}(x,\boldsymbol{y}(x))y'_1(x)+\frac{\partial G}{\partial y_2}(x,\boldsymbol{y}(x))y'_2(x)=0
        \end{cases}$$
        从而
        $$\begin{pmatrix}
            y'_1(x)\\
            y'_2(x)
        \end{pmatrix}=\begin{pmatrix}
                \frac{\partial F}{\partial y_1} & \frac{\partial F}{\partial y_2}\\
                \frac{\partial G}{\partial y_1} & \frac{\partial G}{\partial y_2}
            \end{pmatrix}^{-1}\begin{pmatrix}
                \frac{\partial F}{\partial x}\\
                \frac{\partial G}{\partial x}
            \end{pmatrix}(x,\boldsymbol{y}(x)),x\in O(x_0,\rho)$$
        \end{proof}

        \paragraph{\colorbox{orange!75}{Note}}求导可以达到线性化的目的。

        \paragraph{\colorbox{pink}{例}}课本例题12.4.4。设$\begin{cases}
            y=y(x)\\
            z=z(x)
        \end{cases}$是由方程组
        $\begin{cases}
            z=xf(x+y)\\
            F(x,y,z)=0
        \end{cases}$所确定的向量值隐函数,其中$f$和$F$分别具有连续的导数和偏导数,求$\dfrac{\d z}{\d x}$.\\
        \textbf{解:}两边关于$x$求偏导。

        \paragraph{\colorbox{lime}{$n$元$m$维向量值隐函数定理}}考虑如下方程组:
        $$\begin{cases}
            F_1(x_1,\cdots,x_n,y_1,\cdots,y_m)=0\\
            F_2(x_1,\cdots,x_n,y_1,\cdots,y_m)=0\\
            \cdots\\
            F_m(x_1,\cdots,x_n,y_1,\cdots,y_m)=0
        \end{cases}$$
        记
        $$\frac{\partial(F_1,\cdots,F_m)}{\partial(y_1,\cdots,y_m)}=\begin{vmatrix}
            \frac{\partial F_1}{\partial y_1}&\frac{\partial F_1}{\partial y_2}&\cdots &\frac{\partial F_1}{\partial y_m}\\
            \vdots&\vdots&&\vdots\\
            \frac{\partial F_m}{\partial y_1}&\frac{\partial F_m}{\partial y_2}&\cdots &\frac{\partial F_m}{\partial y_m}
        \end{vmatrix}$$
        称为$F_1,\cdots,F_m$关于$y_1,\cdots,y_m$的Jacobi行列式。引入记号$\boldsymbol{x}=(x_1,\cdots,x_n),\boldsymbol{y}=(y_1,\cdots,y_m)$,$\boldsymbol{F}(\boldsymbol{x},\boldsymbol{y})=\begin{pmatrix}
            F_1(\boldsymbol{x},\boldsymbol{y})\\
            \vdots\\
            F_m(\boldsymbol{x},\boldsymbol{y})
        \end{pmatrix}$
        。于是如果$\boldsymbol{F}$满足:
        \begin{enumerate}[(1)]
            \item $\boldsymbol{F}(\boldsymbol{x}_0,\boldsymbol{y}_0)=0$;\\
            \item 在$D=\{(\boldsymbol{x},\boldsymbol{y}):|x_i-x_i^0|\le a_i,|y_j-y_j^0|\le b_j,i=1,\cdots,n,j=1,\cdots,m\}$连续,而且有连续偏导数;
            \item $\frac{\partial(F_1,\cdots,F_m)}{\partial(y_1,\cdots,y_m)}(\boldsymbol{x}_0,\boldsymbol{y}_0)\not=0$,
        \end{enumerate}
        则$\exists\rho>0,\eta>0$使得:
        \begin{enumerate}[(i)]
            \item $\forall\boldsymbol{x}\in O(\boldsymbol{x}_0,\rho)$,方程$\boldsymbol{F}(\boldsymbol{x},\boldsymbol{y})=0$在$O(\boldsymbol{y}_0,\eta)$中有唯一解,记为$\boldsymbol{y}(\boldsymbol{x})$;
            \item $\boldsymbol{y}(\boldsymbol{x}_0)=\boldsymbol{y}_0$;
            \item $\boldsymbol{y}(\boldsymbol{x})\in C^1(O(\boldsymbol{x}_0,\rho))$,而且$\boldsymbol{y}'(\boldsymbol{x})=-\left(\nabla_{\boldsymbol{y}}\boldsymbol{F}\right)^{-1}\nabla_{\boldsymbol{x}}\boldsymbol{F}(\boldsymbol{x},\boldsymbol{y}(\boldsymbol{x}))$,其中$$\nabla_{\boldsymbol{x}}\boldsymbol{F}=\begin{pmatrix}
                \nabla_{\boldsymbol{x}}F_1\\
                \vdots\\
                \nabla_{\boldsymbol{x}}F_m
            \end{pmatrix}=\begin{pmatrix}
                \frac{\partial F_1}{\partial x_1}&\cdots&\frac{\partial F_1}{\partial x_n}\\
                \frac{\partial F_2}{\partial x_1}&\cdots&\frac{\partial F_2}{\partial x_n}\\
                \vdots&&\vdots\\
                \frac{\partial F_m}{\partial x_1}&\cdots&\frac{\partial F_m}{\partial x_n}\\
            \end{pmatrix}$$
        \end{enumerate}
        \begin{proof}
            参见史济怀/徐森林。
        \end{proof}

        下面我们用线性方程组的角度大致理解条件“Jacobi矩阵行列式不为零”:在$(\boldsymbol{x}_0,\boldsymbol{y}_0)$附近对$\boldsymbol{F}(\boldsymbol{x},\boldsymbol{y})$做Taylor展开:
        $$F_1(\boldsymbol{x},\boldsymbol{y})=F_1(\boldsymbol{x}_0,\boldsymbol{y}_0)+(\boldsymbol{x}-\boldsymbol{x}_0)\nabla_{\boldsymbol{x}}F_1(\boldsymbol{x}_0,\boldsymbol{y}_0)+(\boldsymbol{y}-\boldsymbol{y}_0)\nabla_{\boldsymbol{y}}F_1(\boldsymbol{x}_0,\boldsymbol{y}_0)+h_1(\boldsymbol{x},\boldsymbol{y})$$
        $$\vdots$$
        $$F_m(\boldsymbol{x},\boldsymbol{y})=F_m(\boldsymbol{x}_0,\boldsymbol{y}_0)+(\boldsymbol{x}-\boldsymbol{x}_0)\nabla_{\boldsymbol{x}}F_m(\boldsymbol{x}_0,\boldsymbol{y}_0)+(\boldsymbol{y}-\boldsymbol{y}_0)\nabla_{\boldsymbol{y}}F_m(\boldsymbol{x}_0,\boldsymbol{y}_0)+h_m(\boldsymbol{x},\boldsymbol{y})$$
        从而(因$F_i(\boldsymbol{x}_0,\boldsymbol{y}_0)=0,i=1,\cdots,m$)$$0=\boldsymbol{F(\boldsymbol{x},\boldsymbol{y})}=\nabla_{\boldsymbol{x}}\boldsymbol{F}(\boldsymbol{x}_0,\boldsymbol{y}_0)(\boldsymbol{x}-\boldsymbol{x}_0)+\nabla_{\boldsymbol{y}}\boldsymbol{F}(\boldsymbol{x}_0,\boldsymbol{y}_0)(\boldsymbol{y}-\boldsymbol{y}_0)+\boldsymbol{h}(\boldsymbol{x},\boldsymbol{y}),$$
        于是隐函数存在就等价于
        \begin{equation}
            0=\nabla_{\boldsymbol{x}}\boldsymbol{F}(\boldsymbol{x}_0,\boldsymbol{y}_0)(\boldsymbol{x}-\boldsymbol{x}_0)+\nabla_{\boldsymbol{y}}\boldsymbol{F}(\boldsymbol{x}_0,\boldsymbol{y}_0)(\boldsymbol{y}-\boldsymbol{y}_0)+\boldsymbol{h}(\boldsymbol{x},\boldsymbol{y}),\label{线性方程}
        \end{equation}
        其中
        $$\boldsymbol{h}(\boldsymbol{x},\boldsymbol{y})=\boldsymbol{F(\boldsymbol{x},\boldsymbol{y})}-\nabla_{\boldsymbol{x}}\boldsymbol{F}(\boldsymbol{x}_0,\boldsymbol{y}_0)(\boldsymbol{x}-\boldsymbol{x}_0)-\nabla_{\boldsymbol{y}}\boldsymbol{F}(\boldsymbol{x}_0,\boldsymbol{y}_0)(\boldsymbol{y}-\boldsymbol{y}_0),$$
        方程\eqref{线性方程}的线性近似方程为:
        \begin{equation}
            \nabla_{\boldsymbol{x}}\boldsymbol{F}(\boldsymbol{x}_0,\boldsymbol{y}_0)(\boldsymbol{x}-\boldsymbol{x}_0)+\nabla_{\boldsymbol{y}}\boldsymbol{F}(\boldsymbol{x}_0,\boldsymbol{y}_0)(\boldsymbol{y}-\boldsymbol{y}_0)=0\label{线性}
        \end{equation}
        方程\eqref{线性}对$\forall\boldsymbol{x}\in\mathrm{R}^n$可解$\Leftrightarrow\nabla_{\boldsymbol{y}}\boldsymbol{F}(\boldsymbol{x}_0,\boldsymbol{y}_0)$可逆。

        这里用到了以下(描述不太严谨的)定理:“线性方程可解”+“非线性扰动$\boldsymbol{h}(\boldsymbol{x},\boldsymbol{y})$满足
        $||\boldsymbol{h}(\boldsymbol{x},\boldsymbol{y}_1)-\boldsymbol{h}(\boldsymbol{x},\boldsymbol{y}_2)||\le\varepsilon||\boldsymbol{y}_1-\boldsymbol{y}_2||,(\boldsymbol{x},\boldsymbol{y}_1),(\boldsymbol{x},\boldsymbol{y}_2)$在$(\boldsymbol{x}_0,\boldsymbol{y}_0)$附近”
        $\implies$非线性方程可解。

        \paragraph{\colorbox{orange!75}{Note}}
        \begin{enumerate}
            \item 局部性;
            \item $\frac{\partial(F_1,\cdots,F_m)}{\partial(y_1,\cdots,y_m)}(\boldsymbol{x}_0,\boldsymbol{y}_0)\not=0$只是充分条件;
            \item $C^k-$可微性:若$\boldsymbol{F}\in C^k(D)$,则$\boldsymbol{y}(x)\in C^k(O(\boldsymbol{x}_0,\rho)),k=1,\cdots,k$;若$\boldsymbol{F}\in C^\omega(D)$,则$\boldsymbol{y}(x)\in C^\omega(O(\boldsymbol{x}_0,\rho))$(实解析函数);
            \item 地位对称性,即$x_1,\cdots,x_n,y_1,\cdots,y_m$是平等的。例:如果$\frac{\partial(F_1,\cdots,F_m)}{\partial (x_1,\cdots,x_m)}(\boldsymbol{x}_0,\boldsymbol{y}_0)\not=0$,则$x_1,\cdots,x_m$可表示为$x_{m+1},\cdots,x_n,y_1,\cdots,y_m$的函数。
        \end{enumerate}

    \subsection{逆映射定理}
    \textbf{回顾:}若$f\in C^1((a,b))$且$f'(x_0)\not=0$,则$\exists x_0$的一个邻域$O(x_0,\rho)$和$y_0=f(x_0)$的一个邻域$O(y_0,\eta)$使得
    \begin{enumerate}[(i)]
        \item $f$在$O(x_0,\rho)$上单射,而且$f(O(x_0,\rho))=O(y_0,\eta)$(存在);
        \item 记$g$是$f$在$O(y_0,\eta)$上的反函数,$g\in C^1(O(y_0,\eta))$(连续可微);
        \item $g'(y)=\frac{1}{f'(g(y))},\forall y\in O(y_0,\eta)$.
    \end{enumerate}

    下面考虑多元:若
    $$\begin{cases}
        f_1(x_1,\cdots,x_n)=y_1\\
        \vdots\\
        f_n(x_1,\cdots,x_n)=y_n
    \end{cases}$$
    而且$\boldsymbol{f}(\boldsymbol{x}_0)=\boldsymbol{y}_0$,\textcolor{red}{Q:}在什么条件下,$x_1,\cdots,x_n$可写成$y_1,\cdots,y_n$的函数?

    \paragraph{\colorbox{lime}{局部逆映射定理}}设$D\subset \mathrm{R}^n$是一个区域,$\boldsymbol{x}_0\in D,\boldsymbol{f}:D\rightarrow \mathrm{R}^n$满足以下条件:
    \begin{enumerate}[(1)]
        \item $\boldsymbol{f}\in C^1(D)(\boldsymbol{f}=(f_1,\cdots,f_n))$;
        \item $\p{(f_1,\cdots,f_n)}{(x_1,\cdots,x_n)}(\boldsymbol{x}_0)\not=0;$
    \end{enumerate}
    记$\boldsymbol{y}_0=\boldsymbol{f}(\boldsymbol{x}_0)$,则存在$\boldsymbol{x}_0$的一个邻域$U$以及$\boldsymbol{y}_0$的一个邻域$V$,使得:
    \begin{enumerate}[(i)]
        \item $\boldsymbol{f}$在$U$上是单射,而且$\boldsymbol{f}(U)=V$(此时反函数存在唯一);
        \item 记$\boldsymbol{g}$是$\boldsymbol{f}$在$U$上的逆映射,则$\boldsymbol{g}\in C^1(V)$(连续可微性);
        \item $\boldsymbol{g}'(\boldsymbol{y})=(\boldsymbol{f}'(\boldsymbol{g}(\boldsymbol{y})))^{-1},\forall\boldsymbol{y}\in V$(此时$\boldsymbol{f}\circ\boldsymbol{g}(\boldsymbol{y})=\boldsymbol{y},\forall\boldsymbol{y}\in V;\boldsymbol{g}\circ\boldsymbol{f}(\boldsymbol{x})=\boldsymbol{x},\forall\boldsymbol{x}\in U$)(Jacobi矩阵的逆矩阵)
    \end{enumerate}
    \begin{proof}
        考虑$\boldsymbol{F}(\boldsymbol{x},\boldsymbol{y})=\boldsymbol{f}(\boldsymbol{x})-\boldsymbol{y}(F_i(\boldsymbol{x},\boldsymbol{y})=f_i(\boldsymbol{x})-y_i,i=1,\cdots,n)(D\times \mathrm{R}^n\rightarrow \mathrm{R}^n)$,则$\boldsymbol{f}\in C^1(D\times \mathrm{R}^n)$,而且$\boldsymbol{F}(\boldsymbol{x}_0,\boldsymbol{y}_0)=\boldsymbol{f}(\boldsymbol{x}_0)-\boldsymbol{y}_0=0$,$\p{(f_1,\cdots,f_n)}{(x_1,\cdots,x_n)}(\boldsymbol{x}_0)\not=0$,由隐函数定理知:$\exists\rho>0,\eta>0,$以及唯一的隐函数$\boldsymbol{g}\in C^1(O(\boldsymbol{y}_0,\rho))$,s.t.:
        \begin{enumerate}[(i)]
            \item $\boldsymbol{g}(\boldsymbol{y}_0)=\boldsymbol{x}_0,\boldsymbol{F}(\boldsymbol{g}(y),\boldsymbol{y})=\boldsymbol{f}(\boldsymbol{g}(\boldsymbol{y}))-\boldsymbol{y}=0,\forall\boldsymbol{y}\in O(\boldsymbol{y}_0,\rho),||\boldsymbol{g}(\boldsymbol{y})-\boldsymbol{x}_0||<\eta$.令$V=O(\boldsymbol{y}_0,\rho),U=\boldsymbol{g}(V)$,则$\boldsymbol{g}\in C^1(V)$;
            \item 而且$\boldsymbol{f}(U)=V,\boldsymbol{f}$在$U$上是单射(In fact, if $\boldsymbol{f}(\boldsymbol{x}_1)=\boldsymbol{f}(\boldsymbol{x}_2),\boldsymbol{x}_1,\boldsymbol{x}_2\in U$,则$\exists\boldsymbol{y}_1,\boldsymbol{y}_2\in V$,s.t.$\boldsymbol{g}(\boldsymbol{y}_1)=\boldsymbol{x}_1,\boldsymbol{g}(\boldsymbol{y}_2)=\boldsymbol{x}_2$,则有$\boldsymbol{y}_1=\boldsymbol{f}\circ\boldsymbol{g}(\boldsymbol{y_1})=\boldsymbol{f}\circ\boldsymbol{g}(\boldsymbol{y_2})=\boldsymbol{y}_2\Rightarrow \boldsymbol{x}_1=\boldsymbol{x}_2$.
            \item 下面证明:$U$是开集。Claim:$U=O(\boldsymbol{x}_0,\eta)\cap(\boldsymbol{f})^{-1}(V)$,其中$(\boldsymbol{f})^{-1}(V)$是$V$的原像,即$(\boldsymbol{f})^{-1}(V)=\{\boldsymbol{x}\in D:\boldsymbol{f}(\boldsymbol{x})\in V\}$.首先显然有$U\subset O(\boldsymbol{x}_0,\eta)\cup(\boldsymbol{f})^{-1}(V)$;反之,$\forall\tilde{\boldsymbol{x}}\in O(\boldsymbol{x}_0,\eta)\cup(\boldsymbol{f})^{-1}(V),$有$||\tilde{\boldsymbol{x}}-\boldsymbol{x}_0||<\eta$且$\boldsymbol{f}(\tilde{\boldsymbol{x}})\in V$(因为第一条,有唯一性),故$\boldsymbol{g}\circ\boldsymbol{f}(\tilde{\boldsymbol{x}})\in U$(由唯一性,只能有$\boldsymbol{g}\circ\boldsymbol{f}(\tilde{\boldsymbol{x}})=\tilde{\boldsymbol{x}}$),所以$O(\boldsymbol{x}_0,\eta)\cup(\boldsymbol{f})^{-1}(V)\subset U$.又因为$O(\boldsymbol{x}_0,\eta)$与$(\boldsymbol{f})^{-1}(V)$(连续映射把开集映为开集)都是开集,所以$U$是开集。
            \item 求导方法:由$\boldsymbol{f}(\boldsymbol{g}(\boldsymbol{y}))=\boldsymbol{y},\forall\boldsymbol{y}\in V$,两边关于$y$求导得:
            $$\boldsymbol{f}'(\boldsymbol{g}(\boldsymbol{y}))\boldsymbol{g}'(\boldsymbol{y})=I\text{(单位矩阵)}\Rightarrow \boldsymbol{g}'(\boldsymbol{y})=(\boldsymbol{f}'(\boldsymbol{g}(\boldsymbol{y})))^{-1},\forall\boldsymbol{y}\in V.$$
            \item $C^k$可微性:若$\boldsymbol{f}\in C^k(D),\boldsymbol{f}'(\boldsymbol{x}_0)$可逆,则$\boldsymbol{g}\in C^k(V),k=1,2,\cdots$(这一点容易得到);若$\boldsymbol{f}\in C^\omega(D),\boldsymbol{f}'(\boldsymbol{x}_0)$可逆,则$\boldsymbol{g}\in C^\omega(V)$(这一点不容易证明);
            \item 关于单射的理解:当$\boldsymbol{x},\boldsymbol{y}$在$\boldsymbol{x}_0$附近时,有$$\boldsymbol{f}(\boldsymbol{x})=\boldsymbol{f}(\boldsymbol{x}_0)+\boldsymbol{f}'(\boldsymbol{x}_0)(\boldsymbol{x}-\boldsymbol{x}_0)+\boldsymbol{h}(\boldsymbol{x}),\boldsymbol{h}(\boldsymbol{x})=\boldsymbol{f}(\boldsymbol{x})-\boldsymbol{f}(\boldsymbol{x}_0)-\boldsymbol{f}'(\boldsymbol{x}_0)(\boldsymbol{x}-\boldsymbol{x}_0),$$
            $$\boldsymbol{f}(\boldsymbol{y})=\boldsymbol{f}(\boldsymbol{y}_0)+\boldsymbol{f}'(\boldsymbol{y}_0)(\boldsymbol{y}-\boldsymbol{y}_0)+\boldsymbol{h}(\boldsymbol{y}),\boldsymbol{h}(\boldsymbol{y})=\boldsymbol{f}(\boldsymbol{y})-\boldsymbol{f}(\boldsymbol{y}_0)-\boldsymbol{f}'(\boldsymbol{y}_0)(\boldsymbol{y}-\boldsymbol{y}_0),$$
            两式相减,若$\boldsymbol{f}(\boldsymbol{x})=\boldsymbol{f}(\boldsymbol{y})$,则$$\boldsymbol{f}'(\boldsymbol{x}_0)(\boldsymbol{x}-\boldsymbol{y})=\boldsymbol{h}(\boldsymbol{y})-\boldsymbol{h}(\boldsymbol{x}),$$从而$$\boldsymbol{x}-\boldsymbol{y}=(\boldsymbol{f}'(\boldsymbol{x}_0))^{-1}[\boldsymbol{h}(\boldsymbol{y})-\boldsymbol{h}(\boldsymbol{x})]$$
            $$\Rightarrow||\boldsymbol{x}-\boldsymbol{y}||\le ||(\boldsymbol{f}'(\boldsymbol{x}_0))^{-1}||\cdot||\boldsymbol{h}(\boldsymbol{y})-\boldsymbol{h}(\boldsymbol{x})||,$$类似\ref{无穷小}的理由知此时只能有$\boldsymbol{x}=\boldsymbol{y}$.
        \end{enumerate}
    \end{proof}

    \textbf{回顾}:若$f\in C^1(a,b)$且$f'(x)\not=0,\forall x\in (a,b)$,则$f^{-1}$在$(f(a),f(b))$(or $(f(b),f(a))$)上存在。\textcolor{red}{Q}:若$\boldsymbol{f}\in C^1(D)$且$\p{(f_1,\cdots,f_n)}{(x_1,\cdots,x_n)}(\boldsymbol{x}_0)\not=0,\forall\boldsymbol{x}\in D$,那么$\boldsymbol{f}$是否在$D$上是单射?

    \textcolor{red}{A}:当$n\ge 2$时一般不对。\textbf{例:}
    $$\boldsymbol{f}:\mathbb{R}^2\rightarrow \mathbb{R}^2$$
    $$(x,y)\mapsto(\e^x\cos y,\e^x\sin y)$$%箭头
    不是单射,而且$$\p{(f_1,f_2)}{(x,y)}=\begin{vmatrix}
        \e^x\cos y&-\e^x\sin y\\
        \e^x\sin y&\e^x\cos y
    \end{vmatrix}=\e^{2x}\not=0,\forall(x,y)\in\mathbb{R}^2.$$

    \begin{tcolorbox}[colback=red!5!white,arc=1mm,colframe=red!75!black,fonttitle=\bfseries,title=总结]
        \begin{enumerate}
            \item 一元隐函数定理($\partial_yF(x_0,y_0)\not=0$);
            \item 多元隐函数定理($\partial_yF(\boldsymbol{x}_0,y_0)\not=0$);
            \item 一元二维向量值隐函数定理($\p{(F,G)}{(y_1,y_2)}(x_0,\boldsymbol{y}_0)\not=0$);
            \item n元m维向量值隐函数定理($\p{(F_1,\cdots,F_m)}{(y_1,\cdots,y_m)}(\boldsymbol{x}_0,\boldsymbol{y}_0)\not=0$);
            \item 逆映射定理($\p{(f_1,\cdots,f_n)}{(x_1,\cdots,x_n)}(\boldsymbol{x}_0)\not=0$).
        \end{enumerate}
    \end{tcolorbox}

    \begin{tcbitemize}[raster columns=2,raster equal height,equal height group=AT,colframe=red!75!black,colback=red!5!white,fonttitle=\bfseries]
        \tcbitem[title={遗留问题}]
        \begin{enumerate}[a]
            \item (延拓问题:)隐函数存在区间有多大?
            \item The behavior of Implicit function?
        \end{enumerate}
        \tcbitem[title=要求]
        \begin{enumerate}[a]
            \item 会计算隐函数的导数(隐式微分法);
            \item 掌握定理的条件及结论(注意一下偏导不为零的条件),了解证明。
        \end{enumerate}
    \end{tcbitemize}
    \begin{tcbitemize}[raster columns=2,raster equal height,
        colframe=red!75!black,colback=red!5!white,equal height group=AT,fonttitle=\bfseries]
        \tcbitem[title=理论]
        \begin{enumerate}[1]
            \item 偏导数(方向导数)与全微分的概念,求导方法(+、-、$\times$、$\div$,复合)
            \item 中值定理及Taylor公式
            \item 隐函数定理及逆映射定理
        \end{enumerate}
        \tcbitem[title=应用]
        \begin{enumerate}[1]
            \item 在几何中的应用——微分几何
            \item 极值
        \end{enumerate}
    \end{tcbitemize}



    \subsection{偏导数在几何中的应用}
    \begin{enumerate}
        \item 空间曲线的切线方程与法平面方程:关键是求出切向量;
        \item 曲面的法线方程与法平面方程:关键是求出法向量;
        \item 曲线在交点处的夹角,曲面在交线一点处的夹角。
    \end{enumerate}

    \subsubsection{空间曲线的切向量和法平面}
    \begin{enumerate}
        \item 参数表示
        
        \textbf{切向量与切线方程:}设$(x(t),y(t),z(t)),t\in[a,b]$是$\mathbb{R}^3$中的一条曲线,记为$\Gamma$;而且$x(t),y(t),z(t)\in C^1([a,b]),(x'(t))^2+(y'(t))^2  +(z'(t))^2\not=0,\forall t\in[a,b]$。取一点$P_0(x(t_0),y(t_0),z(t_0))\in\Gamma$,计算$P_0$处的切线方程(切线定义为割线的极限)。设$P(x(t),y(t),z(t))$是曲线上异于$P_0$的一点,则割线$\overline{PP_0}$的方程为:$$\frac{x-x_0}{x(t)-x(t_0)}=\frac{y-y_0}{y(t)-y(t_0)}=\frac{z-z_0}{z(t)-z(t_0)},$$
        同时除以$t-t_0$,有:
        $$\frac{x-x_0}{\frac{x(t)-x(t_0)}{t-t_0}}=\frac{y-y_0}{\frac{y(t)-y(t_0)}{t-t_0}}=\frac{z-z_0}{\frac{z(t)-z(t_0)}{t-t_0}},$$
        令$t\to t_0$,得:
        \begin{equation}
            \frac{x-x_0}{x'(t)}=\frac{y-y_0}{y'(t)}=\frac{z-z_0}{z'(t)},\label{导数}
        \end{equation}
        即为切线方程。说明:若$x'(t_0)=0$,则方程\eqref{导数}变为$\begin{cases}
            x=x_0\\
            \frac{y-y_0}{y'(t)}=\frac{z-z_0}{z'(t)}
        \end{cases}$;若$x'(t_0)=y'(t_0)=0$,则\eqref{导数}变为$\begin{cases}
            x=x_0\\
            y=y_0
        \end{cases}$.
        
        也可这样推导:$\Gamma$在$P_0$点处的切向量为$(x'(t),y'(t),z'(t))$,则$\Gamma$过$P_0$点的切线方程为(切线方程与切向量平行):
        \begin{equation}
            \frac{x-x_0}{x'(t)}=\frac{y-y_0}{y'(t)}=\frac{z-z_0}{z'(t)},
        \end{equation}

        \textbf{法平面方程:}$\Gamma$在$P_0$点处的法平面方程为(直接利用点积为零):
        \begin{equation}
            (x-x_0)x'(t_0)+(y-y_0)y'(t_0)+(z-z_0)z'(t_0)=0.
        \end{equation}

        \item 显示方程
        
        设曲线$\Gamma$由$\begin{cases}
            y=y(x)\\
            z=z(x)
        \end{cases},x\in(a,b)$给出,取$P_0(x_0,y(x_0),z(x_0))\in\Gamma$,则$\Gamma$在$P_0$点的切向量为$(1,y'(x_0),z'(x_0))$,故$\Gamma$在$P_0$点的切线方程为$$x-x_0=\frac{y-y_0}{y'(x_0)}=\frac{z-z_0}{z'(x_0)},$$$\Gamma$在$P_0$点的法平面方程为$$x-x_0+(y-y_0)y'(x_0)+(z-z_0)z'(x_0)=0.$$

        \item 隐式表示
        
        设曲线$\Gamma$由两张曲面相交而成,方程为
        $$\begin{cases}
            F(x,y,z)=0\\
            G(x,y,z)=0
        \end{cases},(x,y,z)\in D,\text{且}F\in C^1(D),G\in C^1(D).$$
        此时设$D_0=(x_0,y_0,z_0)\in\Gamma,$而且$\begin{pmatrix}
            \p{F}{x}(D_0)&\p{F}{y}(D_0)&\p{F}{z}(D_0)\\
            \p{G}{x}(D_0)&\p{G}{y}(D_0)&\p{G}{z}(D_0)
        \end{pmatrix}$满秩,计算$\Gamma$在$P_0$点处的切向量。

        此时不妨设$\p{(F,G)}{(y,z)}(P_0)\not=0$,则由\textcolor{red}{隐函数定理}知$\exists\rho>0$以及$y(x)\in C^1(O(x_0,\rho)),z(x)\in C^1(O(x_0,\rho))$使得$$\begin{cases}
            y(x_0)=y_0\\
            z(x_0)=z_0
        \end{cases}\text{且}\begin{cases}
            F(x,y(x),z(x))=0\\
            G(x,y(x),z(x))=0
        \end{cases},x\in o(x_0,\rho)$$
        因此$\Gamma$在$P_0$点的一个邻域中,$\Gamma$的方程可由$(x,y(x),z(x))$给出,$x\in O(x_0,\rho)$,从而$\Gamma$在$P_0$点的切向量为$(1,y'(x_0),z'(x_0))$,且
        \begin{align*}
            \begin{pmatrix}
                y'(x_0)\\
                z'(x_0)
            \end{pmatrix}&=-\begin{pmatrix}
                \p{F}{y}&\p{F}{z}\\
                \p{G}{y}&\p{G}{z}
            \end{pmatrix}^{-1}(P_0)\begin{pmatrix}
                \p{F}{x}\\
                \p{G}{x}
            \end{pmatrix}(P_0)=-\frac{\begin{pmatrix}
                \p{G}{z}&-\p{F}{z}\\
                -\p{G}{y}&\p{F}{y}
            \end{pmatrix}\begin{pmatrix}
                \p{F}{x}\\
                \p{G}{x}
            \end{pmatrix}(P_0)}{\p{(F,G)}{(y,z)}(P_0)}\\
            &=\begin{pmatrix}
                \frac{\p{(F,G)}{(z,x)}}{\p{(F,G)}{(y,z)}(P_0)}&
                \frac{\p{(F,G)}{(x,y)}}{\p{(F,G)}{(y,z)}(P_0)}
            \end{pmatrix}^T
        \end{align*}
        从而$\Gamma$在$P_0$点的切向量为$$\left(\p{(F,G)}{(y,z)}(P_0),\p{(F,G)}{(z,x)}(P_0),\p{(F,G)}{(x,y)}(P_0)\right).$$

        \textcolor{red}{另一种方法:}设$\Gamma$的参数方程为$(x(t),y(t),z(t))$,而且$P_0(x_0,y_0,z_0)=(x(t_0),y(t_0),z(t_0))$,又因为
        $$\begin{cases}
            F(x(t),y(t),z(t))=0\\
            G(x(t),y(t),z(t))=0
        \end{cases},$$
        关于$t$求偏导后再将$t=t_0$带入,有:
        $$\begin{cases}
            F_x(P_0)x'(t_0)+F_y(P_0)y'(t_0)+F_z(P_0)z'(t_0)=0\\
            G_x(P_0)x'(t_0)+G_y(P_0)y'(t_0)+G_z(P_0)z'(t_0)=0
        \end{cases}$$
        $$\Rightarrow (x'(t_0),y'(t_0),z'(t_0))\perp(F_x(P_0),F_y(P_0),F_z(P_0)),(x'(t_0),y'(t_0),z'(t_0))\perp(G_x(P_0),G_y(P_0),G_z(P_0))$$\label{垂直}
        所以\begin{align*}
            &(x'(t_0),y'(t_0),z'(t_0))//(F_x(P_0),F_y(P_0),F_z(P_0))\times(G_x(P_0),G_y(P_0),G_z(P_0))\\
            =&\begin{vmatrix}
                \boldsymbol{i}&\boldsymbol{j}&\boldsymbol{z}\\
                F_x(P_0)&F_y(P_0)&F_z(P_0)\\
                G_x(P_0)&G_y(P_0)&G_z(P_0)
            \end{vmatrix}=\begin{pmatrix}
                \p{(F,G)}{(y,z)}(P_0),\p{(F,G)}{(z,x)}(P_0),\p{(F,G)}{(x,y)}(P_0)
            \end{pmatrix}
        \end{align*}
    \end{enumerate}

    \subsubsection{曲面的切平面与法线}
    关键是计算法向量

    \begin{enumerate}[(1)]
        \item 隐式方程:设曲面$S$的方程为
        $$F(x,y,z)=0,(x,y,z)\in D\subset\mathbb{R}^2,$$
        而且$F\in C^1(D),F_x^2+F_y^2+F_z^2\not=0,\forall(x,y,z)\in D$.取$P_0(x_0,y_0,z_0)\in S$,欲计算$S$在$P_0$处的切平面方程与法线方程。任取$S$上过点$P_0$的一条曲线$(x(t),y(t),z(t))$,设$P_0(x(t_0),y(t_0),z(t_0))$.显然有
        $$F(x(t),y(t),z(t))=0,t\in O(t_0,\rho)$$与\ref{垂直}同理有
        $$(F_x(P_0),F_y(P_0),F_z(P_0))\perp(x'(t_0),y'(t_0),z'(t_0)),$$
        因此$S$上过$P_0$的所有曲线在$P_0$处的切线落在一个平面$\pi$上,称$\pi$为$S$在$P_0$点的切平面,$\boldsymbol{n}=(F_x(P_0),F_y(P_0),F_z(P_0))$称为$S$在$P_0$点的法向量。所以$S$在$P_0$处的切平面方程为:
        \begin{equation}
            F_x(P_0)(x-x_0)+F_y(P_0)(y-y_0)+F_z(P_0)(z-z_0)=0;
        \end{equation}
        $S$在$P_0$处的法线方程为:
        \begin{equation}
            \frac{x-x_0}{F_x(P_0)}=\frac{y-y_0}{F_y(P_0)}=\frac{z-z_0}{F_z(P_0)}.
        \end{equation}

        \item 显示表示:设曲面$S$的方程为$z=f(x,y),(x,y)\in D\subset\mathrm{R}^2$,令$F(x,y,z)=f(x,y)-z$,则$S$的方程为$F(x,y,z)=0$。从而在$P_0(x_0,y_0,z_0)$处的法向量$\boldsymbol{n}=(F_x(P_0),F_y(P_0),F_z(P_0))=(f_x(P_0),f_y(P_0),-1)$
        
        或:设$(x(t),y(t),z(t))$在$S$上,则$z(t)=f(x(t),y(t))\Rightarrow z'(t_0)=f_x(P_0)x'(t_0)+f_y(P_0)y'(t_0)$;从而$(x'(t_0),y'(t_0),z'(t_0))\perp(f_x(P_0),f_y(P_0),-1)$,从而$(f_x(P_0),f_y(P_0),-1)$是法向量。

        从而$S$在$P_0$处的切平面方程为:
        \begin{equation}
            f_x(P_0)(x-x_0)+f_y(P_0)(y-y_0)-(z-z_0)=0
        \end{equation}
        或可写作
        \begin{align*}
            z=&z_0+f_x(P_0)(x-x_0)+f_y(P_0)(y-y_0)\\
            =&f(x_0,y_0)+f_x(P_0)(x-x_0)+f_y(P_0)(y-y_0)
        \end{align*}
        回忆Taylor展开:
        \begin{equation}
            f(x,y)=f(x_0,y_0)+f_x(P_0)(x-x_0)+f_y(P_0)(y-y_0)+o(\sqrt{(x-x_0)^2+(y-y_0)^2})
        \end{equation}
        从而可微也即可用切平面逼近。$S$在$P_0$处的法线方程为:
        \begin{equation}
            \frac{x-x_0}{f_x(P_0)}=\frac{y-y_0}{f_y(P_0)}=\frac{z-z_0}{-1}.
        \end{equation}

        \item 参数方程:设曲面$S$的参数方程为$\begin{cases}
            x=x(u,v)\\
            y=y(u,v)\\
            z=z(u,v)
        \end{cases},(u,v)\in D\subset\mathrm{R}^2$,而且$\begin{pmatrix}
            \p{x}{u}&\p{x}{v}\\
            \p{y}{u}&\p{y}{v}\\
            \p{z}{u}&\p{z}{v}
        \end{pmatrix}(u,v)$在$D$上满秩,取$U-$曲线$\begin{cases}
            x=x(u,v_0)\\
            y=y(u,v_0)\\
            z=z(u,v_0)
        \end{cases}$,则在$P_0$处的切向量为$$\left(\p{x}{u}(u_0,v_0),\p{y}{u}(u_0,v_0),\p{z}{u}(u_0,v_0)\right);$$取$V-$曲线$\begin{cases}
            x=x(u_0,v)\\
            y=y(u_0,v)\\
            z=z(u_0,v)
        \end{cases}$,则在$P_0$处的切向量为$$\left(\p{x}{v}(u_0,v_0),\p{y}{v}(u_0,v_0),\p{z}{v}(u_0,v_0)\right),$$从而$S$在$P_0(x_0,y_0,z_0)$处的法向量为\begin{align*}
            &\left(\p{x}{u},\p{y}{u},\p{z}{u}\right)(u_0,v_0)\times\left(\p{x}{v},\p{y}{v},\p{z}{v}\right)(u_0,v_0)\\
            =&\begin{vmatrix}
                \boldsymbol{i}&\boldsymbol{j}&\boldsymbol{z}\\
                \p{x}{u}&\p{y}{u}&\p{z}{u}\\
                \p{x}{v}&\p{y}{v}&\p{z}{v}
            \end{vmatrix}(u_0,v_0)=\left(\p{(y,z)}{(u,v)},\p{(z,x)}{(u,v)},\p{(x,y)}{(u,v)}\right).
        \end{align*}

        %%%在这里插入%%%
        \textbf{或利用反函数定理推导:}设曲面$S$的参数方程为$\begin{cases}
            x=x(u,v)\\
            y=y(u,v)\\
            z=z(u,v)
        \end{cases},(u,v)\in D\subset\mathrm{R}^2$,而且$\begin{pmatrix}
            \p{x}{u}&\p{x}{v}\\
            \p{y}{u}&\p{y}{v}\\
            \p{z}{u}&\p{z}{v}
        \end{pmatrix}(u,v)$在$D$上满秩。设$P_0(x_0,y_0,z_0)\in S$,对应于$(u_0,v_0)$(即$\begin{cases}
            x=x(u_0,v_0)\\
            y=y(u_0,v_0)\\
            z=z(u_0,v_0)
        \end{cases}$),不妨设$\begin{vmatrix}
            \p{x}{u}&\p{x}{v}\\
            \p{y}{u}&\p{y}{v}
        \end{vmatrix}(u_0,v_0)\not=0$,于是由逆映射定理知:$\exists\rho>0$以及$\tilde{f},\tilde{g}\in C^1(O((x_0,y_0),\rho))$,s.t.
        $$\begin{cases}
            \tilde{f}(x_0,y_0)=u_0\\
            \tilde{g}(x_0,y_0)=v_0
        \end{cases}\text{以及}\begin{cases}
            x=f(\tilde{f}(x,y),\tilde{g}(x,y))\\
            y=g(\tilde{f}(x,y),\tilde{g}(x,y))
        \end{cases},(x,y)\in O((x_0,y_0),\rho)$$
        这样,曲面$S$在$O((x_0,y_0),\rho)$上有显示表达式$z=h(\tilde{f}(x,y),\tilde{g}(x,y)):=k(x,y)$。从而在$P_0$点的法向量为$\left(\p{k}{x}(x_0,y_0),\p{k}{y}(x_0,y_0),-1\right)$。又因为
        $$\p{k}{x}(x_0,y_0)=\p{h}{u}(u_0,v_0)\p{\tilde{f}}{x}(x_0,y_0)+\p{h}{v}(u_0,v_0)\p{\tilde{g}}{x}(x_0,y_0)$$
        $$\p{k}{y}(x_0,y_0)=\p{h}{u}(u_0,v_0)\p{\tilde{f}}{y}(x_0,y_0)+\p{h}{v}(u_0,v_0)\p{\tilde{g}}{y}(x_0,y_0)$$
        从而
        \begin{align*}
            \left(\p{k}{x}(x_0,y_0),\p{k}{y}(x_0,y_0)\right)&=\left(\p{h}{u}(u_0,v_0),\p{h}{v}(u_0,v_0)\right)\begin{pmatrix}
                \p{\tilde{f}}{x}(x_0,y_0)&\p{\tilde{f}}{y}(x_0,y_0)\\
                \p{\tilde{g}}{x}(x_0,y_0)&\p{\tilde{g}}{y}(x_0,y_0)
            \end{pmatrix}\\
            &=\left(\p{h}{u}(u_0,v_0),\p{h}{v}(u_0,v_0)\right)\begin{pmatrix}
                \p{\tilde{f}}{u}(u_0,v_0)& \p{\tilde{f}}{v}(u_0,v_0)\\
                \p{\tilde{g}}{u}(u_0,v_0)& \p{\tilde{g}}{v}(u_0,v_0)
            \end{pmatrix}^{-1}\\
            &=\left(\p{h}{u}(u_0,v_0),\p{h}{v}(u_0,v_0)\right)\frac{\begin{pmatrix}
                \p{\tilde{g}}{v}(u_0,v_0)&-\p{\tilde{f}}{v}(u_0,v_0)\\
                -\p{\tilde{g}}{u}(u_0,v_0)&\p{\tilde{f}}{u}(u_0,v_0)
            \end{pmatrix}}{\p{(f,g)}{(u,v)}(u_0,v_0)}
        \end{align*}
        所以在$P_0$处的法向量为$\left(\p{(g,h)}{(u,v)}(u_0,v_0),\p{(h,f)}{(u,v)}(u_0,v_0),\p{(f,g)}{(u,v)}(u_0,v_0)\right)$.






        \subsubsection{计算夹角}
        \begin{enumerate}[(1)]
            \item 两条曲线在交点处的夹角是指两条曲线在交点处切向量的夹角。
            
            设两条曲线$\Gamma_1,\Gamma_2$在$P_0$处相交,而且在$P_0$点的切向量分别为$\boldsymbol{\tau}_1,\boldsymbol{\tau}_2$,则夹角$\alpha$满足$\cos\alpha=\frac{\boldsymbol{\tau}_1\cdot\boldsymbol{\tau}_2}{||\boldsymbol{\tau}_1||\cdot||\boldsymbol{\tau}_2||}$.
            \item 两张曲面在交线处上一点处的夹角是指它们在这点处的法向量的夹角。
            
            \paragraph{\colorbox{pink}{例}}圆柱面$x^2+y^2=a^2$与马鞍面$bz=xy$的夹角。

            解:设$(x_0,y_0,z_0)$是交线上的一点,则圆柱面在$P_0$处的法向量为$(2x_0,2y_0,0)$,马鞍面在$P_0$处的法向量为$(y_0,x_0,-b)$。所以在$P_0$处的夹角$\alpha$满足$\cos\alpha=\frac{4x_0y_0}{\sqrt{4(x_0^2+y_0^2)}\sqrt{x_0^2+y_0^2+b^2}}=\frac{4x_0y_0}{a\sqrt{a^2+b^2}}$
        \end{enumerate}
    \end{enumerate}



    \subsection{条件极值——(最)优化问题}
    \paragraph{\colorbox{lime}{定义}}设$\Omega\in\mathbb{R}^n$是一个区域,$\boldsymbol{x}_0\in\Omega,f(\boldsymbol{x})$是$\Omega$上的一个函数。如果$\exists r>0$,s.t.$O(\boldsymbol{x}_0,r)\subset\Omega$,而且$f(\boldsymbol{x})\ge f(\boldsymbol{x}_0),\forall\boldsymbol{x}\in O(\boldsymbol{x}_0,r)$,则称$\boldsymbol{x}_0$是$f$在$\Omega$上的极小值点,$f(\boldsymbol{x}_0)$称为相应的极小值;如果“$f(\boldsymbol{x})\ge f(\boldsymbol{x}_0),\forall\boldsymbol{x}\in O(\boldsymbol{x}_0,r)$”换成“$f(\boldsymbol{x})> f(\boldsymbol{x}_0),\forall\boldsymbol{x}\in O(\boldsymbol{x}_0,r)\backslash\{\boldsymbol{x}_0\}$”,则称$\boldsymbol{x}_0$为$f$在$\Omega$上的一个严格极小(大)值点,相应的$f(\boldsymbol{x}_0)$称为严格极小(大)值。“严格极小(大)值点”“严格极小(大)值”等定义类似。

    \paragraph{\textcolor{red}{Q:}}如何来求极值与极值点?\\
    先来看必要条件:

    回顾一元函数:\begin{enumerate}
        \item 若$f\in C^1(a,b),x_0\in(a,b)$是极值点,则$f'(x_0)=0$;
        \item 若$f\in C^2(a,b),x_0\in(a,b)$是极小值点,则$f''(x_0)\ge 0$;若$x_0$是极大值点,则$f''(x_0)\le 0$;
    \end{enumerate}

    \paragraph{\colorbox{lime}{定理}}($n$元函数最值的必要条件)  若$\Omega\subset\mathbb{R}^n$是一个区域,$\boldsymbol{x}_0\in\Omega,f\in C^1(\Omega)$
    \begin{enumerate}[(1)]
        \item 如果$\boldsymbol{x}_0$是$f$在$\Omega$上的一个极值点,则$\nabla f(\boldsymbol{x}_0)=0$(即$\p{f}{x_i}(\boldsymbol{x}_0)=0,i=1,2,\cdots,n$);
        \item 若$f\in C^2(\Omega),\boldsymbol{x}_0$是$f$在$\Omega$上的一个极大值点,则Hesse矩阵$\mathrm{H}f(\boldsymbol{x}_0)=\left(\frac{\partial^2f}{\partial x_i\partial x_j}(\boldsymbol{x}_0)\right)\le 0$,即$\mathrm{H}f(\boldsymbol{x}_0)$是半负定矩阵;反之若为极小值点,则$\mathrm{H}f(\boldsymbol{x}_0)\ge 0$,即$\mathrm{H}f(\boldsymbol{x}_0)$是半正定矩阵。
    \end{enumerate}
    \begin{proof}
        令$g(t)=f(\boldsymbol{x}_0+t\boldsymbol{h})$,其中$|t|<\rho,\boldsymbol{h}\in S^{n-1}$(单位球面),则$g(t)\in C^1(-\rho,\rho)$,而且$g(t)\ge g(0),\forall t\in(-\rho,\rho)$,所以$g'(0)=0\Rightarrow \boldsymbol{h}\nabla f(\boldsymbol{x}_0)=0,\forall\boldsymbol{h}\in S^{n-1}$,从而$\nabla f(\boldsymbol{x}_0)=0$.

        (2)的证明:\textcolor{blue}{方法1:}
        若$f\in C^2(\Omega),$则$g\in C^2(-\rho,\rho)$;又当$\boldsymbol{x}_0$是$f$在$\Omega$上的极大值点时,$0$是$g$在$(-\rho,\rho)$上的极大值点,所以$g''(0)\le 0$。而
        $$g'(t)=\boldsymbol{h}\nabla f(\boldsymbol{x}_0+t\boldsymbol{h})=\sum_{i=1}^nh_i\p{f}{x_0}(\boldsymbol{x}_0+t\boldsymbol{h})$$
        $$g''(t)=\sum_{i=1}^nh_i\left(\sum_{j=1}^nh_j\frac{\partial^2f}{\partial x_i\partial x_j}(\boldsymbol{x}_0+t\boldsymbol{h})\right)=\sum_{i,j=1}^nh_ih_j\frac{\partial^2f}{\partial x_i\partial x_j}(\boldsymbol{x}_0+t\boldsymbol{h})$$
        所以$$g''(0)=\sum_{i,j=1}^nh_ih_j\frac{\partial^2f}{\partial x_i\partial x_j}(\boldsymbol{x}_0)=\boldsymbol{h}\cdot \mathrm{H}f(\boldsymbol{x}_0)\boldsymbol{h}^T\le 0,\forall\boldsymbol{h}\in S^{n-1}$$
        所以$\mathrm{H}f(\boldsymbol{x}_0)\le 0$.

        \textcolor{blue}{方法2:}若$\boldsymbol{x}_0$是$f$在$\Omega$上的一个极小值点,则$f(\boldsymbol{x}_0+\varepsilon\boldsymbol{h})\ge f(\boldsymbol{x}_0),\forall|\varepsilon|<\rho,\boldsymbol{h}\in S^{n-1}$。又因为
        $$f(\boldsymbol{x}_0+\varepsilon\boldsymbol{h})=f(\boldsymbol{x}_0)+\nabla f(\boldsymbol{x}_0)\cdot\varepsilon\boldsymbol{h}+\frac{1}{2}(\varepsilon\boldsymbol{h})\mathrm{H}f(\boldsymbol{x}_0)(\varepsilon\boldsymbol{h})^T+o(\varepsilon^2)=f(\boldsymbol{x}_0)+\frac{\varepsilon^2}{2}\boldsymbol{h}\mathrm{H}f(\boldsymbol{x}_0)\boldsymbol{h}^T+o(\varepsilon^2)$$
        我们有
        $$\frac{\varepsilon^2}{2}\boldsymbol{h}\mathrm{H}f(\boldsymbol{x}_0)\boldsymbol{h}^T+o(\varepsilon^2)\ge 0,\forall|\varepsilon|<\rho,
        \boldsymbol{h}\in S^{n-1}$$
        $$\Rightarrow\frac{1}{2}\boldsymbol{h}\mathrm{H}f(\boldsymbol{x}_0)\boldsymbol{h}^T+\frac{o(\varepsilon^2)}{\varepsilon^2}\ge 0,\forall\boldsymbol{h}\in S^{n-1}$$
        令$\varepsilon\to 0$,得$\boldsymbol{h}\mathrm{H}f(\boldsymbol{x}_0)\boldsymbol{h}^T\ge 0,\forall\boldsymbol{h}\in S^{n-1},$所以$\mathrm{H}f(\boldsymbol{x}_0)\ge 0$.
    \end{proof}

    \paragraph{\colorbox{pink}{例}}若$\begin{cases}
        \frac{\partial^2f}{\partial x^2}+\frac{\partial^2f}{\partial y^2}-f=0,x^2+y^2<1,\\
        f|_{x^2+y^2=1}=0
    \end{cases}$,则$f\equiv 0$.
    \begin{proof}
        令$\Omega=\{(x,y):x^2+y^2<1\}$.若$\exists(x_0,y_0)\in \Omega,s.t.f(x_0,y_0)=\max_{\Omega}|f|>0$,则$$\begin{pmatrix}
            \frac{\partial^2f}{\partial x^2}&\frac{\partial^2f}{\partial x\partial y}\\
            \frac{\partial^2f}{\partial x\partial y}&\frac{\partial^2f}{\partial y^2}
        \end{pmatrix}(x_0,y_0)\le 0\text{(Hesse矩阵半负定)}\Rightarrow \left(\frac{\partial^2f}{\partial x^2}+\frac{\partial^2f}{\partial y^2}\right)(x_0,y_0)\le 0\text{(}Tr(\mathrm{H})\le 0\text{)}$$
        从而$\left(\frac{\partial^2f}{\partial x^2}+\frac{\partial^2f}{\partial y^2}\right)(x_0,y_0)-f(x_0,y_0)<0$,矛盾。
    \end{proof}

    \paragraph{\colorbox{orange!75}{Note}}\begin{enumerate}
        \item 只是必要条件。例:$f(x,y)=xy$,则$\nabla f(0,0)=(0,0)$,但是$(0,0)$不是$f$的极值点。又例:$f(x,y)=x^3+y^3$,则$\nabla f(0,0)=(0,0),\mathrm{H}f(0,0)=\begin{pmatrix}
            0&0\\
            0&0
        \end{pmatrix}$,但$(0,0)$不是极值点。又例:$f(x,y)=4xy^2+x^2+y^4$,此时有$\nabla f(0,0)=(0,0),\mathrm{H}f(0,0)=\begin{pmatrix}
            2&0\\
            0&0
        \end{pmatrix}\ge 0$,但是$(0,0)$不是极值点。
        \item 偏导数不存在的点也可能是极值点。例:$f(x,y)=|x|+|y|$,$(0,0)$为其极小值点,但$f$在$(0,0)$处偏导数不存在。
    \end{enumerate}

    回顾一元函数:设$f\in C^2((a,b)),x_0\in(a,b),f'(x_0)=0$,
    \begin{enumerate}
        \item 若$f''(x_0)>0$,则$x_0$是$f$的严格极小值点;
        \item 若$f''(x_0)<0$,则$x_0$是$f$的严格极大值点;
        \item 若$f''(x_0)=0$,无法判断,这时需要计算高阶导数。更一般地,若$k\ge 2$且$f'(x_0)=\cdots=f^{(k-1)}(x_0)=0,f^{(k)}(x_0)\not=0$,则
        \begin{enumerate}
            \item 如果$k$是奇数,则$x_0$不是极值;
            \item 如果$k$是偶数,则
            \begin{itemize}
                \item $f^{(k)}(x_0)>0$,则$x_0$是$f$的极小值点;
                \item $f^{(k)}(x_0)<0$,则$x_0$是$f$的极大值点
            \end{itemize}
        \end{enumerate}
        (若$f'(x_0)=f''(x_0)=0,f^{(3)}(x_0)\not=0$,则$f(x_0+h)-f(x_0)=\frac{1}{3!}f^{(3)}(x_0)h^3+o(h^3)$,因为$h$可正可负,故无法根据$f^{(3)}(x_0)$的符号来判断右边的正负。)
    \end{enumerate}
    \paragraph{\colorbox{lime}{定理}}($n$元函数取极值的充分条件) (由于多元函数的高阶导数比较复杂,我们只考虑2阶导数。)设$f\in C^2(D),\boldsymbol{x}_0\in D$且$\nabla f(\boldsymbol{x}_0)=0$,那么
    \begin{enumerate}[1)]
        \item $\mathrm{H}f(\boldsymbol{x}_0)>0$(正定矩阵),则$\boldsymbol{x}_0$是严格极小值;
        \item $\mathrm{H}f(\boldsymbol{x}_0)<0$(负定矩阵),则$\boldsymbol{x}_0$是严格极大值;
        \item $\mathrm{H}f(\boldsymbol{x}_0)$有正的特征值,也有负的特征值(不定矩阵),则$\x_0$不是极值点。
        \item 其他情况(即$\mathrm{H}f(\x_0)$的特征值$\lambda_i\ge 0$且有$0$特征值,或$\mathrm{H}f(\x_0)$的特征值$\lambda_i\le 0$且有$0$特征值),无法判断$\x_0$是否为极值点。
    \end{enumerate}
    \begin{proof}
        \begin{enumerate}[1)]
            \item 若$\mathrm{H}f(\boldsymbol{x}_0)>0$,需要证明:$f(\x_0+\varepsilon\boldsymbol{h})>f(\x_0),0<\varepsilon<\rho,\boldsymbol{h}\in S^{n-1}$.因为
            \begin{align*}
                f(\x_0+\varepsilon\boldsymbol{h})&=f(\x_0)+\frac{1}{2}(\varepsilon\boldsymbol{h})\mathrm{H}f(\x_0)(\varepsilon\boldsymbol{h})^T+o(\varepsilon^2)\\
                &=f(\x_0)+\frac{\varepsilon^2}{2}\boldsymbol{h}\mathrm{H}f(\x_0)\boldsymbol{h}^T+o(\varepsilon^2)
            \end{align*}
            从而只需说明$\frac{\varepsilon^2}{2}\boldsymbol{h}\mathrm{H}f(\x_0)\boldsymbol{h}^T+o(\varepsilon^2)\ge 0$.由于$\mathrm{H}f(\x_0)>0,$所以$\lambda_{\max}\ge \boldsymbol{h}\mathrm{H}f(\x_0)\boldsymbol{h}^T\ge \lambda_{\min}$(最小特征值)(验证利用实对称矩阵特征向量正交),$\forall\boldsymbol{h}\in S^{n-1}$,所以$\exists\eta>0,s.t.$ $$\frac{\varepsilon^2}{2}\boldsymbol{h}\mathrm{H}f(\x_0)\boldsymbol{h}^T+o(\varepsilon^2)\ge \frac{\lambda_{min}}{2}\varepsilon^2+o(\varepsilon^2)>0,\text{当}0<|\varepsilon|<\eta$$
            从而$f(\x_0+\varepsilon\boldsymbol{h})>f(\x_0),0<|\varepsilon|<\eta,\boldsymbol{h}\in S^{n-1}$.
            \item 证明与1)类似。
            \item 设$\lambda$是$\mathrm{H}f(\x_0)$的一个正特征值,$\boldsymbol{h}^T\in S^{n-1}$是相应的特征向量(即$\mathrm{H}f(\x_0)\boldsymbol{h}^T=\lambda\boldsymbol{h}^T$),则
            \begin{align*}
                f(\x_0+\varepsilon\boldsymbol{h})=&f(\x_0)+\frac{\varepsilon^2}{2}\boldsymbol{h}\mathrm{H}f(\x_0)\boldsymbol{h}^T+o(\varepsilon^2)\\
                =&f(\x_0)+\frac{\lambda}{2}\varepsilon^2+o(\varepsilon^2)>f(\x_0),\text{当}0<|\varepsilon|\text{且充分小}.
            \end{align*}
            同理,若$\mu$是$\mathrm{H}f(\x_0)$的一个负特征值,$\boldsymbol{k}^T\in S^{n-1}$是相应的特征向量,则$f(\x_0+\varepsilon\boldsymbol{k})<f(\x_0)$(当$0<|\varepsilon|$且充分小)。所以$\x_0$不是$f$的极值点。
            \item 例如:$f(x,y)=4xy^2+x^2+y^4$,则$\nabla f(0,0)=(0,0),\mathrm{H}f(0,0)=\begin{pmatrix}
                2&0\\
                0&0
            \end{pmatrix}$,特征值为$2$和$0$,但$(0,0)$不是极值。$f(x,y)=x^2+y^4$,则$\nabla f(0,0)=(0,0),\mathrm{H}f(0,0)=\begin{pmatrix}
                2&0\\
                0&0
            \end{pmatrix}$,这时$(0,0)$是极小值。
        \end{enumerate}
    \end{proof}

    \paragraph{\colorbox{lime}{例}}$f(x,y)=x^2-2xy^2+y^4-y^5$的极值。\\
    \textbf{解:}找$(x_0,y_0)$使$\begin{cases}
        \p{f}{x}(x_0,y_0)=0\\
        \p{f}{y}(x_0,y_0)=0
    \end{cases}\Rightarrow (x_0,y_0)=(0,0)$,$\nabla^2f(0,0)=\begin{pmatrix}
        2&0\\
        0&0
    \end{pmatrix}$,特征值是2和0,分别对应特征向量$\begin{pmatrix}
        1\\
        0
    \end{pmatrix}$和$\begin{pmatrix}
        0\\
        1
    \end{pmatrix}$。在0特征值对应的特征向量附近找点(例如课本上找$f(\varepsilon^2,\varepsilon)$以及$f(\varepsilon^2,-\varepsilon)$)

    \paragraph{\colorbox{lime}{推论}}设$(x_0,y_0)\in D\subset\mathbb{R}^2,f\in C^2(D)$且$\nabla f(x_0,y_0)=(0,0),A=\frac{\partial^2f}{\partial x^2}(x_0,y_0),B=\frac{\partial^2f}{\partial x\partial y}(x_0,y_0),C=\frac{\partial^2f}{\partial y^2}(x_0,y_0)$
    \begin{enumerate}
        \item 当$A>0$且$AC-B^2>0$时(行列式大于零)(Hesse矩阵正定),$(x_0,y_0)$是极小值点;
        \item 当$A<0$且$AC-B^2>0$时,$(x_0,y_0)$是极大值点;
        \item $AC-B^2<0$:$(x_0,y_0)$不是极值点;
        \item $AC-B^2=0$:无法判断$(x_0,y_0)$是否为极值点。
    \end{enumerate}

    \begin{tcolorbox}[colback=red!5!white,arc=1mm,colframe=red!75!black,fonttitle=\bfseries,title=总结:求多元极值的步骤]
        设$f$是一个$n$元函数
    \begin{enumerate}[1)]
        \item 解方程$\p{f}{x_i}(\x_0)=0,i=1,2,\cdots,n$,找出驻点(临界点);
        \item 计算$\mathrm{H}f(\x_0)=\frac{\partial^2f}{\partial x_i\partial x_j}(\x_0)$,应用上述定理;
        \item 寻找$\p{f}{x_i}$不存在的点,并判断其是否为极值点.
    \end{enumerate}
    \end{tcolorbox}


    \subsubsection{多元凹(凸)函数}
    \paragraph{\colorbox{lime}{定义}}设$D\subset\mathrm{R}^2$是一个凸集,$f$是$D$上的一个函数。如果
    \begin{enumerate}[(i)]
        \item $\forall\x,\y\in D,\theta\in(0,1)$都有$f(\theta\x+(1-\theta)\y)\le \theta f(\x)+(1-\theta)f(\y)$,则称$f$是$D$上的凸函数;
        \item $\forall\x,\y\in D$且$\x\not=\y$,$\theta\in(0,1)$都有$f(\theta\x+(1-\theta)\y)<\theta f(\x)+(1-\theta)f(\y)$,则称$f$是$D$上的严格凸函数。
    \end{enumerate}

    同理可定义(严格)凹函数。

    \paragraph{\colorbox{lime}{定理}}设$D\subset\mathrm{R}^2$是一个凸集,$f\in C^2(D)$,则如下定价:
    \begin{enumerate}[1)]
        \item $f$是$D$上的凸函数;
        \item $\forall\x_0\in D$,都有$f(\x)\ge f(\x_0)+\nabla f(\x_0)(\x-\x_0),\forall\x\in D$(也即$f(\x)$在$\x_0$点的切线/切平面上方);
        \item $\forall\x\in D$,有$\mathrm{H}f(\x_0)=\left(\frac{\partial^2f}{\partial x_i\partial x_j}(\x)\right)\ge 0$
    \end{enumerate}
    \begin{proof}
        1)$\Rightarrow $2):作为练习;

        2)$\Rightarrow $3):给定$\x_0\in D,0<|\varepsilon|$充分小;于是$\forall\boldsymbol{h}\in S^{n-1}:$ 
        $$f(\x_0+\varepsilon\boldsymbol{h})\ge f(\x_0)+\nabla f(\x_0)\cdot(\varepsilon\boldsymbol{h})$$
        又$$f(\x_0+\varepsilon\boldsymbol{h})=f(\x_0)+\nabla f(\x_0)\cdot(\varepsilon\boldsymbol{h})+\frac{1}{2}\varepsilon^2\boldsymbol{h}\mathrm{H}f(\x_0)\boldsymbol{h}^T+o(\varepsilon^2)$$
        所以$\frac{1}{2}\varepsilon^2\boldsymbol{h}\mathrm{H}f(\x_0)\boldsymbol{h}^T+o(\varepsilon^2)\ge 0$,类似有$\boldsymbol{h}\mathrm{H}f(\x_0)\boldsymbol{h}^T\ge 0,\forall\boldsymbol{h}\in S^{n-1}\Rightarrow \mathrm{H}f(\x_0)\ge 0$.

        3)$\Rightarrow $1)任取$\x,\y\in D,\theta\in (0,1)$,令$\boldsymbol{\xi}=\theta\x+(1-\theta)\y$,要证$f(\boldsymbol{\xi})\le\theta f(\x)+(1-\theta)f(\y)$,我们有:
        \begin{align}
            f(\x)=&f(\boldsymbol{\xi})+\nabla f(\boldsymbol{\xi})(\xi-\boldsymbol{\xi})+\frac{1}{2}(\xi-\boldsymbol{\xi})\mathrm{H}f(\eta\x+(1-\eta)\boldsymbol{\xi})(\xi-\boldsymbol{\xi})^T (\eta\in(0,1))\notag\\
            \ge&f(\boldsymbol{\xi})+\nabla f(\boldsymbol{\xi})(\x-\boldsymbol{\xi})\label{x}
        \end{align}
        同理
        \begin{equation}
            f(\y)\ge f(\boldsymbol{\xi})+\nabla f(\boldsymbol{\xi})(\y-\boldsymbol{\xi})\label{y}
        \end{equation}
        \eqref{x}$\times\theta+$\eqref{y}$\times(1-\theta)$,有
        $$\theta f(\x)+(1-\theta)f(\y)\ge f(\boldsymbol{\xi})=f(\theta\x+(1-\theta)\y).$$
    \end{proof}


    \subsubsection{函数的最值}
    设$D\in\mathbb{R}^n$是一个区域,
    \begin{enumerate}[{Case} 1{:}]
        \item $D$是有界区域;
        \item $D=\mathbb{R}^n$.
    \end{enumerate}
    \vspace{2ex}
    \paragraph{\textcolor{blue}{Case 1}}: $D$是有界区域,$f:\bar{D}\to \mathbb{R}$的连续函数,$f\in C^1(D)$,则$f$在$\bar{D}$上的最小值和最大值都可以取到。
    \paragraph{\textcolor{red}{Q:}}如何找$f$的最值?
    \paragraph{回顾一元函数:}$f\in C([a,b]),f\in C^1((a,b))$,
    \begin{enumerate}[{Step} 1{:}]
        \item 找出$f$在$(a,b)$内的所有驻点,即找出所有满足$f'(x_0)=0$的$x_0$,计算$f(x_0)$;
        \item 计算$f(a),f(b)$;
        \item 比较$f(a),f(b)$和所有$f(x_0),x_0\in(a,b)$是驻点。
    \end{enumerate}
    \paragraph{高维时:}
    \begin{enumerate}[{Step} 1{:}]
        \item 找出所有满足$\nabla f(\x_0)=0,\x_0\in D$的点$\x_0$,记为$\x_1,\cdots,\x_n$,计算$f(\x_i),i=1,2,\cdots,n$;
        \item 计算$f$在$\partial D$上的最大值和最小值,分别记为$A,B$;
        \item $f_{\max} =\max\{A,f(\x_1),\cdots,f(\x_n)\};f_{\min} =\min\{A,f(\x_1),\cdots,f(\x_n)\};$
    \end{enumerate}
    \paragraph{\colorbox{orange!75}{Note}}如果根据实际情况能过判断出最值不在边界上,只需计算$f$的所有驻点,并比较其值即可。
    
    \vspace{2ex}
    \paragraph{\textcolor{blue}{Case 2}}:$D\in \mathbb{R}^n,f\in C^1(R^n)$,则函数的最值不一定存在.但以下几种情况可判断最值存在:
    \begin{enumerate}[(i)]
        \item 若$\underline{\lim}_{||x||\to+\infty}f(\x)=+\infty$,则$f(\x)$在$\mathbb{R}^n$上一定可以取到最小值;
        \item 若$\overline{\lim}_{||x||\to+\infty}f(\x)=-\infty$,则$f(\x)$在$\mathbb{R}^n$上一定可以取到最大值。
    \end{enumerate}
    若$f$满足(i),我们来找最小值。做法:找出$f(\x)$在$\mathbb{R}^n$上的全部驻点$\{\x_i\}_{i=1}^n$,计算$f(\x_i)$,则最小值为$\min\{f(\x_1),\cdots,f(\x_n)\}$.

    \paragraph{\colorbox{pink}{例}}Guass最小二乘法:
    $$Q(a,b)=\sum_{i=1}^n(ax_i+b-y_i)^2,a,b\in\mathbb{R}.$$
    \textbf{解:}
    \begin{enumerate}[{Step} 1{:}]
        \item $\lim_{||a||+||b||\to+\infty}Q(a,b)=+\infty$(要求$x_i$不全相等);
        \item 令$\begin{cases}
            \p{Q}{a}(a,b)=0\\
            \p{Q}{b}(a,b)=0
        \end{cases}$得到$\begin{cases}
            a=\\
            b=
        \end{cases}$,从而得到最小值为$Q(a,b)$.
    \end{enumerate}

    
    \subsubsection{条件极值与Lagrange乘数法}
    \paragraph{\colorbox{pink}{例}}
    \begin{enumerate}[(1)]
        \item 在有界区域上求函数的最值时,需要计算$f$在$\partial D$上的最值,这就是一个条件极值的问题;
        \item 设$\Gamma\subset\mathbb{R}^3$是一条闭曲线,计算$\Gamma$到原点的距离,即$d(\boldsymbol{O},\Gamma)=\inf_{\x\in\Gamma}dist(\boldsymbol{O},\x)=$\\$\min_{\x\in\Gamma}\sqrt{x^2+y^2+z^2},\x=(x,y,z)$,即我们要计算当$(x,y,z)$限制在$\Gamma$上时,求$f(x,y,z)=\sqrt{x^2+y^2+z^2}$的最小值。
    \end{enumerate}
    \paragraph{条件极值的提法}设$D\subset\mathbb{R}^n$是一个区域,$f(\x)\in C^1(D),g_1(D),\cdots,g_m(D)\in C^1(D),m<n$.令$S=\{\x\in D:g_1(\x)=0,\cdots,g_m(\x)=0\}$,假设$\forall\x\in S,\begin{pmatrix}
        \p{g_1}{x_1}&\cdots&\p{g_1}{x_n}\\
        \vdots&&\vdots\\
        \p{g_m}{x_1}&\cdots&\p{g_m}{x_n}
    \end{pmatrix}_{m\times n}(\x)$是满秩的($f$称为目标函数,$g_1(\x)=0,\cdots,g_m(\x)=0(*)$称为约束条件).设$\x_0\in S$,如果$\exists\rho>0,s.t.\forall\x\in O(\x_0,\rho)\cap S\subset D$,都有$f(\x)\ge f(\x_0)$,则称$\x_0$是$f(\x)$在约束条件(*)之下的条件极小值点,$f(\x_0)$称为$f$在约束条件(*)下的极小值.也可定义严格条件极值点和严格条件极值.

    \paragraph{\textcolor{blue}{Goal:}}找$\x_0$是条件极值点的充分条件.\\
    为此,我们首先来找必要条件.从最简单的情况开始:目标函数:$f(x,y)$;约束条件:$g(x,y)=0$.

    几何观察:
    %%%插图
    由图知,在极值点$P_0$和$M_0$处,$\nabla f(P_0)//\nabla g(P_0),\nabla f(M_0)//\nabla g(M_0)$.\\
    \fbox{Ex:}目标函数为$f(x,y,z)$,约束条件为$g(x,y,z)=0$.
    
    \paragraph{\colorbox{lime}{定理}}($f(x,y)$在约束条件$g(x,y)=0$下取条件极值的必要条件) 设$D\subset\mathbb{R}^2$是一个区域,$f,g\in C^1(D)$.若$(x_0,y_0)$是$f$在约束条件$g(x,y)=0$之下的条件极值点,则$\exists\lambda_0\in\mathbb{R},s.t.\nabla f(x_0,y_0)=\lambda\nabla g(x_0,y_0)$;进一步,如果$f,g\in C^2(D)$,则:
    \begin{enumerate}[i)]
        \item 若$(x_0,y_0)$是条件极小值点,则$\forall(a,b)\in\mathbb{R}^2$且$(a,b)\cdot\nabla g(x_0,y_0)=0$,都有$(a,b)\mathrm{H}(f-\lambda_0g)(x_0,y_0)\begin{pmatrix}
            a\\
            b
        \end{pmatrix}\ge 0$;
        \item 若$(x_0,y_0)$是条件极大值点,则$\forall(a,b)\in\mathbb{R}^2$且$(a,b)\cdot\nabla g(x_0,y_0)=0$,都有$(a,b)\mathrm{H}(f-\lambda_0g)(x_0,y_0)\begin{pmatrix}
            a\\
            b
        \end{pmatrix}\le 0$.
    \end{enumerate}
    \begin{proof}
        \textcolor{blue}{方法一(隐函数定理):}基本思想是转化成无条件极值.由于$\nabla g(x_0,y_0)\not=0$,不妨设$\p{g}{y}(x_0,y_0)\not=0$,根据隐函数定理知,$\exists\rho>0$以及$y(x)\in C^1(O(x_0,\rho)),s.t.\begin{cases}
            y(x_0)=y_0,g(x,y(x))=0,\forall x\in O(x_0,\rho)\\
            y'(x)=-\p{g}{x}(x,y(x))/\p{g}{y}(x,y(x))
        \end{cases}$.令$h(x)=f(x,y(x)),x\in O(x_0,\rho)$,则$h(x_0)$是$h(x)$在$O(x_0,\rho)$上的极值,所以$h'(x_0)=0\Rightarrow\p{f}{x}(x,y(x))+\p{f}{y}(x,y(x))y'(x)|_{x=x_0}=0$,所以
        $$\p{f}{x}(x,y(x))-\frac{\p{f}{y}(x_0,y_0)}{\p{g}{y}(x_0,y_0)\p{g}{x}}(x_0,y_0)=0,$$
        令$\lambda_0=\frac{\p{f}{y}(x_0,y_0)}{\p{g}{y}(x_0,y_0)},$则$\nabla f(x_0,y_0)=\lambda_0\nabla g(x_0,y_0)$.

        若$(x_0,y_0)$是极小值点,则$x_0$是$h(x)$在$O(x_0,\rho)$上的极小值点,所以$h''(x_0)\ge 0$.
        \begin{align*}
            h''(x)=&\frac{\partial^2f}{\partial x^2}(x,y(x))+2\frac{\partial^2f}{\partial x\partial y}(x,y(x))y'(x)+\frac{\partial^2f}{\partial y^2}(x,y(x))(y'(x_0))^2\\
            &+\p{f}{y}(x,y(x))y''(x)
        \end{align*}
        \begin{align*}
            \Rightarrow 0\le& \frac{\partial^2f}{\partial x^2}(x_0,y_0)+2\frac{\partial^2f}{\partial x\partial y}(x_0,y_0)y'(x)+\frac{\partial^2f}{\partial y^2}(x_0,y_0)(y'(x_0))^2\\
            &+\p{f}{y}(x_0,y_0)y''(x_0)\\
            =&(1,g'(x_0))\mathrm{H}f(x_0,y_0)\begin{pmatrix}
                1\\
                y'(x_0)
            \end{pmatrix}+\p{f}{y}(x_0,y_0)y''(x_0)
        \end{align*}
        又因为$g(x,y(x))=0,\forall x\in O(x_0,\rho)$,求二阶导数,令$x=y_0$得
        \begin{equation}
            0=(1,g'(x_0))\mathrm{H}f(x_0,y_0)\begin{pmatrix}
                1\\
                y'(x_0)
            \end{pmatrix}+\p{g}{y}(x_0,y_0)y''(x_0) \label{梯度g}
        \end{equation}
        $\lambda_0\times$\eqref{梯度g},两式相减得:$$0\le (1,g'(x_0))\mathrm{H}(f-\lambda_0g)\begin{pmatrix}
            1\\
            y'(x_0)
        \end{pmatrix},$$
        因为$\forall(a,b)\subset\mathbb{R}^2$且$(a,b)\cdot\nabla g(x_0,y_0),$有$$(a,b)\mathrm{H}(f-\lambda_0g)(x_0,y_0)\begin{pmatrix}
            a\\
            b
        \end{pmatrix}\le 0.$$
        \textcolor{blue}{方法2(参数法):}设曲线$g(x,y)=0$在$(x_0,y_0)$附近的参数方程为$((x(t),y(t)),t\in O(t_0,\rho),(x_0,y_0)=(x(t_0),y(t_0))$,则$g(x(t),y(t))=0,t\in O(t_0,\rho)$.令$h(t)=f(x(t),y(t)),t\in O(t_0,\rho)$,则$t_0$是$h(t)$在$O(t_0,\rho)$上的一个极值点,所以$h'(t_0)=0,$即$\nabla f(x_0,y_0)\cdot(x'(t_0),y'(t_0))=0$;又$g(x(t),y(t))=0,\forall t\in O(x_0,\rho)$,所以$\nabla g(x_0,y_0)(x'(t_0),y'(t_0))=0$.从而$\exists\lambda_0\in\mathbb{R},s.t.\nabla f(x_0,y_0)=\lambda_0\nabla g(x_0,y_0)$.

        当$(x_0,y_0)$是条件极大值点时,$h''(t_0)\le 0.$又$h'(t)=\nabla f(x,y)\cdot(x'(t),y'(t))$,所以$h''(t)=(x'(t),y'(x))\mathrm{H}f(x,y)\begin{pmatrix}
            x'(t)\\
            y'(x)
        \end{pmatrix}+\nabla f(x,y)\cdot(x''(t),y''(x)$
        所以$$(x'(t_0),y'(x_0))\mathrm{H}f(x_0,y_0)\begin{pmatrix}
            x'(t_0)\\
            y'(x_0)
        \end{pmatrix}+\nabla f(x_0,y_0)\cdot(x''(t_0),y''(x_0))\le 0.$$
        又因为$$(x'(t_0),y'(x_0))\mathrm{H}g(x_0,y_0)\begin{pmatrix}
            x'(t_0)\\
            y'(x_0)
        \end{pmatrix}+\nabla g(x_0,y_0)\cdot(x''(t_0),y''(x_0))=0,$$
        同理有$$(x'(t_0),y'(x_0))\mathrm{H}(f-\lambda_0g)(x_0,y_0)\begin{pmatrix}
            x'(t_0)\\
            y'(x_0)
        \end{pmatrix}\le 0.$$
        \fbox{Ex:}目标函数$f(x,y,z)=0$,约束条件$\begin{pmatrix}
            H(x,y,z)=0\\
            G(x,y,z)=0
        \end{pmatrix}$
    \end{proof}

    \paragraph{\colorbox{lime}{定理}}($f(\x)$在约束条件(*)下取条件极值的必要条件) 若$\x_0$是$f(\x)$在约束条件(*)下的条件极值,则$\exists\lambda_1,\cdots,\lambda_m\in\mathbb{R},s.t.\nabla f(\x_0)=\sum_{i=1}^m\lambda_i\nabla g_i(\x_0)$.进一步,若$f\in C^2(D),g_i\in C^2(D)$,则
    \begin{enumerate}[i)]
        \item 当$\x_0$是$f$在(*)下的条件极小值点时,有:$\forall\boldsymbol{a}\in span\{\nabla g_1(P_0),\cdots,\nabla g_m(P_0)\}^\perp$,都成立$\boldsymbol{a}\mathrm{H}(f-\sum_{i=1}^n\lambda_ig_i)(\x_0)\boldsymbol{a}^\perp\ge 0$;
        \item 当$\x_0$是$f$在(*)下的条件极大值点时,有:$\forall\boldsymbol{a}\in span\{\nabla g_1(P_0),\cdots,\nabla g_m(P_0)\}^\perp$,都成立$\boldsymbol{a}\mathrm{H}(f-\sum_{i=1}^n\lambda_ig_i)(\x_0)\boldsymbol{a}^\perp\le 0$.
    \end{enumerate}
    \begin{proof}
        任取$S$上过$\x_0$的一条光滑曲线$\Gamma$,其参数方程为$\x=\x(t),t\in O(t_0,\rho),\x_0=\x(t_0)$,则$g_i(\x(t))=0,\forall t\in O(t_0,\rho),i=1,2,\cdots$.又由假设知,$h(t):=f(\x(t))$在$t_0$处取得极值,故$\nabla f(\x_0)\cdot\x'(t_0)=0$;又因$g_i(\x(t))=0,\forall t\in O(t_0,\rho)$,故有$\nabla g_i(\x_0)\cdot\x'(t_0)=0$.由于$\x(t)$的任意性以及$S$是一个$n-m$维的曲面,所以$\nabla f(\x_0),\nabla g_i(\x_0)$都属于同一个$m$维线性空间;又因为$\nabla g_i(\x_0)$是$m$个线性无关的向量,故$\exists!\lambda_1,\cdots,\lambda_m\in\mathbb{R},s.t.\nabla f(\x_0)=\sum_{i=1}^m\lambda_i\nabla g_i(\x_0)$.

        ii)若$\x_0$是$f$在(*)下的条件极小值,则$h(t)$在$t_0$处取得极小值,所以$h''(t_0)\ge 0$.我们下面计算$h''(t)$:
        $$h'(t)=\nabla f(\x_0)\cdot\x'(t)=\sum_{j=1}^n\p{f}{x_j}(\x(t))x'_j(t),$$
        \begin{align*}
            h''(t)=&\sum_{j=1}^n\left[\p{f}{x_j}(\x(t))\right]'x'(t)+\sum_{j=1}^n\p{f}{x_j}(\x(t))x''_j(t)\\
            =&\sum_{j,m=1}^n\frac{\partial^2f}{\partial x_j\partial x_m}x'_m(t)x'_j(t)+\sum_{j=1}^n\p{f}{x_j}(\x(t))x''_j(t)\\
            =&\x'(t)\mathrm{H}f(\x(t))(\x'(t))^T+\nabla f(\x(t))\x''(t)
        \end{align*}
        所以$$\x'(t_0)\mathrm{H}f(\x_0)(\x'(t_0))^T+\nabla f(\x(t_0))\x''(t_0)\ge 0,$$
        又对$g_i(\x(t))=0$求二阶导数,令$t=t_0$,得:
        $$\x'(t_0)\mathrm{H}g(\x(t_0))(\x'(t_0))^T+\nabla g(\x(t_0))\x''(t_0)=0,$$
        从而
        $$\x'(t_0)\mathrm{H}(f-\sum_{j=1}^m\lambda_ig_i)(\x(t_0))(\x'(t_0))^T\ge 0,$$
        由$\x(t)$的任意性知,$\forall\boldsymbol{\alpha}\in span\{\nabla g_1(\x_0),\cdots,g_m(\x_0)\}^\perp$,都有$$\boldsymbol{\alpha}\mathrm{H}(f-\sum_{j=1}^m\lambda_ig_i)(\x(t_0))\boldsymbol{\alpha}^T\ge 0.$$
    \end{proof}
    \fbox{Ex:}用隐函数定理证明

    \paragraph{\colorbox{lime}{定理}}($f$在(*)下取条件极值的充分条件) 设$\x_0\in S$,而且$\exists\lambda_1,\cdots,\lambda_m\in \mathbb{R},s.t.$
    $$\nabla f(\x_0)=\sum_{i=1}^n\lambda_i\nabla g_i(\x_0),$$
    \begin{enumerate}[i)]
        \item 若$\forall\boldsymbol{\alpha}\in span\{\nabla g_1(\x_0),\cdots,g_m(\x_0)\}^\perp\backslash\{\boldsymbol{o}\}$,都有
        $$\boldsymbol{\alpha}\mathrm{H}(f-\sum_{j=1}^m\lambda_ig_i)(\x_0)\boldsymbol{\alpha}^T>0,$$则$\x_0$是严格条件极小值.特别地,如果$\mathrm{H}(f-\sum_{j=1}^m\lambda_ig_i)(\x_0)>0$,则$\x_0$是严格条件极小值;
        \item 若$\forall\boldsymbol{\alpha}\in span\{\nabla g_1(\x_0),\cdots,g_m(\x_0)\}^\perp\backslash\{\boldsymbol{o}\}$,都有
        $$\boldsymbol{\alpha}\mathrm{H}(f-\sum_{j=1}^m\lambda_ig_i)(\x(t_0))\boldsymbol{\alpha}^T<0,$$则$\x_0$是严格条件极大值;特别地,如果$\mathrm{H}(f-\sum_{j=1}^m\lambda_ig_i)(\x_0)<0$,则$\x_0$是严格条件极大值;
        \item 当$m\le n-2$时,若$\exists\boldsymbol{a},\boldsymbol{b}\in span\{\nabla g_1(P_0),\cdots,g_m(P_0)\}^\perp,s.t.$
        $$\boldsymbol{a}\mathrm{H}(f-\sum_{j=1}^m\lambda_ig_i)(\x_0)\boldsymbol{a}^T>0,\quad\boldsymbol{b}\mathrm{H}(f-\sum_{j=1}^m\lambda_ig_i)(\x_0)\boldsymbol{b}^T<0,$$则$\x_0$不是条件极值;
        \item 其余情况(即$\forall\boldsymbol{a}\in span\{\nabla g_1(\x_0),\cdots,g_m(\x_0)\}^\perp$,都有$\boldsymbol{a}\mathrm{H}(f-\sum_{j=1}^m\lambda_ig_i)(\x_0)\boldsymbol{a}^T\ge 0$(或$\le 0$),而且$\exists \boldsymbol{a}_0\in span\{\nabla g_1(\x_0),\cdots,g_m(\x_0)\}^\perp,s.t. \boldsymbol{a}_0\mathrm{H}(f-\sum_{j=1}^m\lambda_ig_i)(\x_0)\boldsymbol{a}_0^T=0$),这种方法不能判断(需要用定义).
        \begin{proof}
            任取$S$上过$\x_0$的一条光滑曲线$\Gamma:x=x(t),t\in O(t_0,\rho),
            \x_0=\x(t_0)$.令$h(t)=f(\x(t)),g_i(\x(t))=0(\nabla g_i(\x(t))\cdot\x'(t))$,则$$h'(t_0)=\nabla f(\x_0)\cdot\x'(t_0)=\left(\nabla f-\sum_{i=1}^m\lambda_i\nabla g_i\right)(\x_0)\cdot\x'(t_0)=\left(\nabla f(\x_0)-\sum_{i=1}^m\lambda_i\nabla g_i(\x_0)\right)\x'(t_0)=0,$$
            $$h''(t_0)=\x'(t_0)\mathrm{H}(f-\sum_{j=1}^m\lambda_ig_i)(\x(t_0))(\x'(t_0))^T>0,$$
            所以$\exists\tilde{\rho}>0,s.t.h(t)>h(t_0),\forall t\in O(t_0,\tilde{\rho})\backslash\{t_0\}$
        \end{proof}
    \end{enumerate}

    \begin{tcolorbox}[colback=red!5!white,arc=1mm,colframe=red!75!black,fonttitle=\bfseries,title=总结:求条件极值的步骤]
        \begin{enumerate}[{Step }1]
            \item 构造Lagrange函数$L(\x,\boldsymbol{\lambda})=f(\x)-\sum_{i=1}^m\lambda_ig_i(\x)$;
            \item 令$\begin{cases}
                \p{L}{x_j}(\x,\boldsymbol{\lambda})=0,j=1,2,\cdots\\
                g_i(\x)=0,i=1,2,\cdots
            \end{cases}$,解出$(\x_0,\boldsymbol{\lambda}_0)$;
            \item 计算$\mathrm{H}(f-\sum_{i=1}^m\lambda_ig_i)(\x_0)$,来判断是否正定;如果$\mathrm{H}(f-\sum_{i=1}^m\lambda_ig_i)(\x_0)$不定,则需要计算:
            $$A(\boldsymbol{a})=\boldsymbol{a}\mathrm{H}(f-\sum_{i=1}^m\lambda_ig_i)(\x_0)\boldsymbol{a}^T,$$
            其中$\boldsymbol{a}\in span\{\nabla g_1(\x_0),\cdots,g_m(\x_0)\}^\perp$:
            \begin{enumerate}[1)]
                \item 若$A(\boldsymbol{a})>0,\forall\boldsymbol{a}\in span\{\nabla g_1(\x_0),\cdots,g_m(\x_0)\}^\perp\backslash\{\boldsymbol{o}\}$,则$\x_0$是条件极小值;
                \item 若$A(\boldsymbol{a})<0,\forall\boldsymbol{a}\in span\{\nabla g_1(\x_0),\cdots,g_m(\x_0)\}^\perp\backslash\{\boldsymbol{o}\}$,则$\x_0$是条件极大值;
                \item 若$\exists\boldsymbol{a},\boldsymbol{b}\in span\{g_1(\x_0),\cdots,g_m(\x_0)\}^\perp\backslash\{\boldsymbol{o}\},s.t.A(\boldsymbol{a})>0,A(\boldsymbol{b})<0$,则$\x_0$不是条件极值.
            \end{enumerate}
        \end{enumerate}
    \end{tcolorbox}
 

    \paragraph{\colorbox{pink}{例}}$\begin{cases}
        f(x,y,z)=x^2+y^2+z^2\\
        g(x,y,z)=0
    \end{cases}$

    \textbf{解:}\begin{enumerate}[(1)]
        \item $L(x,y,z,\lambda)=f(x,y,z)-\lambda g(x,y,z)$;
        \item 令$\begin{cases}
            \p{L}{x}=\p{L}{y}=\p{L}{z}=0\\
            g(x,y,z)=0
        \end{cases}\Rightarrow(x,y,z,\lambda)=(0,0,0,0)$;
        \item $\mathrm{H}L(0,0,0,0)=\begin{pmatrix}
            2&0&0\\
            0&2&0\\
            0&0&-2
        \end{pmatrix}$为不定矩阵,无法判断;
        \item $\{\nabla g(0,0,0,0)\}^T=\{(0,0,1)\}^T=\{(a,b,c):a,b,c\in\mathbb{R}\}$,且$$(a,b,c)\begin{pmatrix}
            2&0&0\\
            0&2&0\\
            0&0&-2
        \end{pmatrix}\begin{pmatrix}
            a\\
            b\\
            c
        \end{pmatrix}=2(a^2+b^2)>0,(a,b,c)\not=(0,0,0),$$
        所以$(0,0,0)$为条件极小值.
    \end{enumerate}

    \subsubsection{条件最值}
    目标函数$f(\x)$;约束条件:(*)$g_i(\x)=0,i=1,2,\cdots,m,m<n$.令$S=\{\x\in D:g_1(\x)=\cdots=g_m(\x)=0\}$.\textcolor{red}{Q:}如何计算$f(\x)$在(*)下的条件最值?即$f(\x)$在$S$上的最值?

    \paragraph{\textcolor{blue}{步骤:}}
    \begin{enumerate}[{Step }1:]
        \item 判断$f$在(*)下的条件极值存在.
        \begin{enumerate}[{Case }1]
            \item 当$S$是紧集时,则条件最大,条件最小都存在(课本例12.7.2,12.7.3,12.7.4);
            \item 当$S$不是紧集时,例:12.7.1,以及直线$\begin{cases}
                x+y+z=1\\
                x+2y+3z=6
            \end{cases}$到原点的距离.一般需要研究清楚$f$在$\partial S$附近的变化情况.
        \end{enumerate}
        \item 找出所有的可能的条件极值点.构造Lagrange函数$L(\x,\boldsymbol{\lambda})=f(\x)-\sum_{i=1}^m\lambda_ig_i(\x)$.令$\begin{cases}
            \p{L}{x_i}(\x_0,\boldsymbol{\lambda}_0)=0,&i=1,2,\cdots,n\\
            g_i(\x)=0,&j=1,2,\cdots,m
        \end{cases}$,解出所有可能极值点,不妨记为$\{\x_k\}_{k=1}^p$;
        \item 计算$f(\x_k),k=1,2,\cdots,p$,则$f$在(*)下的条件极大值为$\max\{f(\x_1),\cdots,f(\x_p)\}$,条件极小值为$\min\{f(\x_1),\cdots,f(\x_p)\}$.
    \end{enumerate}
    \textbf{\textcolor{red}{Remark:}}不需要计算二阶导数.

    \paragraph{\colorbox{pink}{例}}12.7.4:$f(x_1,\cdots,x_n)=x_1^{a_1}\cdots x_n^{a_n},a_i>0,x_i>0$且$\sum_{i=1}^nx_i=a$,求$f(x_1,\cdots,x_n)$在约束条件$\sum_{i=1}^nx_i=a,x_i>0$下的最大值.\\
    \textbf{解:}令$S=\{(x_1,\cdots,x_n)\in \mathbb{R}^n:\sum_{i=1}^nx_i=a,x_i>0\}$.由于$f(\x)>0,\forall\x\in S$且$\lim_{\substack{\x\in S\\ \x\to\partial S}}f(\x)=0$,所以$f(\x)$在$S$上可以取到最大值.$取L(x_1,\cdots,x_n,\lambda)=x_1^{a_1}\cdots x_n^{a_n}-\lambda(x_1+\cdots+x_n-a)$,令$\begin{cases}
        \p{L}{x_i}=\frac{a_i}{x_i}x_1^{a_1}\cdots x_n^{a_n}-\lambda=0\\
        x_1+\cdots+x_n=a
    \end{cases},(x_1,\cdots,x_n)\in S$,所以$\frac{a_1}{x_1}=\cdots=\frac{a_n}{x_n}$,所以$x_i=\frac{aa_i}{a_1+\cdots+a_n},i=1,\cdots,n$,所以$f$在(*)下的条件最大值为(自己算去).

    \paragraph{\colorbox{pink}{例}}12.7.2:$f(x,y,z)=xy+2xz+2yz,S=\{(x,y,z):x>0,y>0,z>0,xyz=a\}$,是一个无界集合.\textcolor{red}{Claim:}$\lim_{\substack{(x,y,z)\in S\\ x+y+z\to+\infty}}f(x,y,z)=+\infty$.当$x+y+z\to+\infty$且$(x,y,z)\in S$时,一定有$x\to+\infty$或者$y\to+\infty$或者$z\to+\infty$,不妨设$x\to+\infty$;又$xyz=a$,有$y\to 0^+$或者$z\to0^+$.当
    $y\to 0^+$时,$xz\to+\infty$,所以$f(x,y,z)\to+\infty$;同理有其他情况.所以$f(x,y,z)$在(*)下一定存在条件最小值.\\
    取$L(z,y,z,\lambda)=xy+2xz+2yz-\lambda(xyz-a)$,令$\begin{cases}
        \p{L}{x}=0\\
        \p{L}{y}=0\\
        \p{L}{z}=0\\
        xyz=a
    \end{cases}$得$\begin{cases}
        x=\sqrt[3]{2a}\\
        y=\sqrt[3]{2a}\\
        z=\frac{\sqrt[3]{2a}}{2}
    \end{cases}$.

    \paragraph{\colorbox{pink}{例}}12.7.3:$f$在有界光滑闭区域$D$上的最值问题.步骤:
    \begin{enumerate}[{Step }1]
        \item 最值存在;
        \item 计算$f$在$D^o$($D$内部)的所有驻点,记为$\{\x_i\}_{i=1}^m$,用Lagrange乘数法计算$f$在$\partial D$上的所有可能的条件极值,记$\{\y_i\}_{j=1}^l$;
        \item 比较$\{f(\x_i),f(\y_j),i=1,2,\cdots,m,j=1,2,\cdots,l\}$,取出最大和最小即可.
    \end{enumerate}
    \textcolor{blue}{方法1:}解出$(x_0,y_0)$,并计算$f(x_0,y_0)$;\textcolor{blue}{方法2:}$f(x_0,y_0)=ax_0^2+2b^2x_0y_0-cy_0^2=\lambda_0$,故只需计算$\lambda_0$的值
    (小技巧:$f$是2次齐次函数,则$x\p{f}{x}+y\p{f}{y}=2f$).
    还可根据方程一定有非零解$\Leftrightarrow$系数行列式$=0$缩小$\lambda_0$的范围.

    \begin{tcolorbox}[colback=red!5!white,arc=1mm,colframe=red!75!black,fonttitle=\bfseries,title=总结:多元函数的极值理论]
    \begin{enumerate}[1)]
        \item 无条件极值:背景与概念(现实生活中的最值)$\rightarrow$取极值的必要条件(一阶导数和二阶导数)$\rightarrow$取极值的充分条件(一阶导数和二阶导数)$\rightarrow$最值问题;
        \item 条件极值:背景与概念(曲线到原点的距离)$\rightarrow$取条件极值的必要条件(几何解释,一阶导数与二阶导数)$\rightarrow$条件极值的充分条件(一阶导数与二阶导数)$\rightarrow$条件最值.
    \end{enumerate}
    \end{tcolorbox}


    \section{重积分}
    Consider the following questions:
    \paragraph{\colorbox{orange!75}{例}}(几何问题:曲顶柱体的体积)设$D\subset\mathbb{R}^2$是一个区域,$f(x,y)$是$D$上的一个非负连续函数,记$\Omega=\{(x,y,z):(x,y)\in D,0\le z\le f(x,y)\}$,称之为曲顶柱体.
    
    下面计算$\Omega$的体积:首先设$D$的一个划分$\Delta=\{\Delta D_1,\cdots,\Delta D_n\},(\Delta D_i)^o\cap(\Delta D_j)^o=\phi(i\not=j),\Delta D_i$是闭集,而且$\cup_{i=1}^{m}{\Delta D_i}=D$.取$(\xi_i,\eta_i)\in\Delta D_i$,计算$\sum_{i=1}^mf(\xi_i,\eta_i)\Delta\sigma_i$,其中$\Delta\sigma_i$是$\Delta D_i$的面积.我们可以认为$\lim_{\lambda(\Delta)\to 0}\sum_{i=1}^mf(\xi_i,\eta_i)\Delta\sigma_i$为$\Omega$的体积,其中$\lambda(\Delta)=\max_{1\le i\le m}diam(\Delta D_i),diam(\Delta D_i)=\sup_{\x,\y\in\Delta D_i}||\x-\y||$.

    \paragraph{\colorbox{orange!75}{例}}(物理问题:已知物体的密度,计算其质量)设物体在$\mathbb{R}^3$中所占的区域为$D$,已知密度为$\rho(x,y,z)$,计算其质量.

    下面计算其质量:取$D$的一个划分$\Delta=\{\Delta D_1,\cdots,\Delta D_n\},(\Delta D_i)^o\cap(\Delta D_j)^o=\phi(i\not=j),\Delta D_i$是闭集,而且$\cup_{i=1}^{m}{\Delta D_i}=D$.取$(\xi_i,\eta_i,\varphi_i)\in\Delta D_i$,计算$\sum_{i=1}^m\rho(\xi_i,\eta_i,\varphi_i)\Delta V_i$(这里$\Delta V_i$表示体积),可以认为$\lim_{\lambda(\Delta)\to 0}\sum_{i=1}^m\rho(\xi_i,\eta_i,\varphi_i)\Delta V_i$是物体的质量.

    遇到的问题:$\Delta D_i$的面积如何求??即使用平行与坐标轴的直线来划分,但与$\partial D$相交时$\Delta D_i$可能很复杂,它的面积如何求解?

    $\mathbb{R}^2$上有界子集的面积:\\
    设$D\subset\mathbb{R}^2$是一个有界集,而且$D\subset U=[a,b]\times[c,d]$,我们来定义它的面积:

    回顾定积分:计算曲边梯形的面积,基本思想:$\begin{cases}
        \text{以直代曲}\\
        \text{内外夹逼}
    \end{cases}$.
    
    取$[a,b]$的一个划分$a=x_0<\cdots<x_n=b$,$[c,d]$的一个划分$c=y_0<\cdots<y_m=d$,设$u_{ij}=[x_{i-1},x_i]\times[y_{j-1},y_j],i=1,\cdots,n,j=1,\cdots,m$,称$\{u_{ij}\}$为$u$的一个划分,记为$T$,令$m^-(D,T)=\sum_{u_{ij}\in D^o}|u_{ij}|,m^+(D,T)=\sum_{u_{ij}\cap\bar{D}\not=\phi}|u_{ij}|$.下面来研究$m^-(D,T)$和$m^+(D,T)$的性质:
    \begin{enumerate}[1)]
        \item 划分加细,$m^-(D,T)$不减,$m^+(D,T)$不增,即如果$\tilde{T}$是$T$的一个加细,则$m^-(D,\tilde{T})\ge m^-(D,T),$\\$m^+(D,\tilde{T})\le m^+(D,T)$.
        \begin{proof}
            设在$[x_0,x_1]$之间插入一个分点$\xi$之后得到的划分为$T^1$,如果$u_{0j}\subset D^o$而且被分点$\xi$分为两部分$u^1_{0j},u^2_{0j}$,则$u^1_{0j},u^2_{0j}\subset D^o$,而且$|u_{0j}|=|u^1_{0j}|+|u^2_{0j}|$,所以$m^-(D,T^1)\ge m^-(D,T)=\sum_{u_{ij}\in D^o}|u_{ij}|$.若$u_{0j}\not\subset D^o$,但$u^1_{0j}\subset D^o$或者$u^2_{0j}\subset D^o$,这时$m^-(D,T^1)> m^-(D,T)$.同理可证$m^+(D,T^1)\le m^+(D,T)$.对任意的划分$T,\tilde{T}$,都可以由有限步``单点分割''构成,所以$m^-(D,\tilde{T})\ge m^-(D,T)$,$m^+(D,\tilde{T})\le m^+(D,T)$.
        \end{proof}
        \item 对$U$的任意两个划分$T,\tilde{T}$,都有$m^-(D,T)\le m^+(D,\tilde{T})$.
        \begin{proof}
            $m^-(D,T)\le m^-(D,T\cup\tilde{T})\le m^+(D,T\cup\tilde{T}) \le m^+(D,\tilde{T})$.
        \end{proof}
        \item 有界性:$\forall$划分$T$,$0\le m^-(D,T)\le (b-a)(c-d),0\le m^+(D,T)\le (b-a)(c-d)$.令$m^-(D)=\sup\{m^-(D):T\text{是}U\text{的划分}\}$,$m^+(D)=\inf\{m^+(D):T\text{是}U\text{的划分}\}$,定义:设$D\subset\mathbb{R}^2$是有界集,如果$m^-(D)=m^+(D)$,则称$D$是可求面积的,而且称$m^-(D)$是它的面积,记为$m(D)$.
        
        定义的合理性:\begin{enumerate}
            \item $D$的可求和性与$m(D)$和$U$的选取无关;
            \item 对于曲边梯形$A=\{(x,y):a\le x\le b,0\le y\le f(x)\}$需要说明$m(A)=\int_a^bf(x)\d x$(eg.13.1.1)
        \end{enumerate}
    \end{enumerate}

    \textcolor{red}{Q:}\begin{enumerate}
        \item 如何来判断一个集合$D$是可求面积的?
        \item 如何计算面积?
    \end{enumerate}

        
    \begin{enumerate}[I)]
        \item 平面上有界点集的面积:
        设$D\subset\mathbb{R}^2$,有界,$D\subset U=[a,b]\times[c,d]$,$T$是$U$的一个划分,$m^-(D,T),m^+(D,T)$,令$m^-(D)=\sup_Tm^-(D,T),m^+(D)=\inf_Tm^+(D,T),m^-(D)\le m^+(D)$,
        \begin{enumerate}[$\cdot$)]
            \item 当$m^-(D)=m^+(D)$时,称$D$是可求面积的.定义的合理性:
            \begin{enumerate}[1)]
                \item $D$是否可求面积以及$m(D)$与$U$的选取关系.取$U_1=[a_1,b_1]\times[c_1,d_1],U_2=[a_2,b_2]\times[c_2,d_2]$且$D\subset U_1,D\subset U_2$.假设用$U_2$覆盖$D$时,$D$是可求面积的,而且面积为$m_2$,需要证明,用$U_1$覆盖$D$时,$D$是可求面积的,而且面积也是$m_2.\forall\varepsilon>0,$存在$U_2$的一个划分$T$,s.t.$m_2^+(D,T)<m_2+\varepsilon,m_2^-(D,T)>m_2-\varepsilon,(m_2=\sup_Tm^-(D,T)=\inf_Tm^+(D,T))$.将$a_1,b_1,c_1,d_1$插入到划分$T$中构成一个$U_1$的划分,记为$T'$,则$m_1^-(D,T')\ge m_1^-(D,T)=\sum_{\substack{U_{ij}\subset D^o\\ U_{ij}\in T}}|U_{ij}|>m_2-\varepsilon,$所以$m_1^-(D)>m_2-\varepsilon$
                当$\partial D\cap \{(a_2,y):y\in[c_2,d_2]\}=\phi$时,平凡;当$\partial D\cap \{(a_2,y):y\in[c_2,d_2]\}\not=\phi$时,将$a_1,a_2-\frac{\varepsilon}{d_1-c_1},c_1,d_1$插入到划分$T$中构成$U_1$的一个划分,记为$\tilde{T}$(右边类似处理),则$m_1^+(D,\tilde{T})\le \frac{\varepsilon}{d_1-c_1}\cdot(d_1-c_1)+m_2^+(D,T)<\varepsilon+m_2+\varepsilon=m_2+2\varepsilon$,即$m_1^+(D,\tilde{T})<m_2+2\varepsilon$,由$\varepsilon$的任意性知$m_1^+(D)=m_1^-(D)=m_2$,所以用$U_1$覆盖$D$时,$D$是可求面积的,而且面积是$m_2$.(\textcolor{red}{Note:}我们证明的大致思路为:欲证$a\le b$,只需证明$\forall\varepsilon>0,a<b+\varepsilon$).
                \item 当$f(x)\ge 0$在$[a,b]$上可积时,$D=\{(x,y):a\le x\le b,0\le y\le f(x)\}$是可求面积的,而且$m(D)=\int_a^bf(x)\d x$
                \begin{proof}
                    令$m=\sup_{[a,b]}f(x)$,则$D\subset[a,b]\times[0,M]$.任取$[a,b]$的一个划分$a=x_0<\cdots<x_n=b$.令$m_i=\inf_{[m_{i-1},m_i]}f(x),M_i=\sup_{[m_{i-1},m_i]}f(x)$,将$\{m_i,M_i\}_{i=1}^n$插入到$[0,M]$得到$[0,M]$的一个划分.由$[a,b],[0,M]$的划分构成$[a,b]\times[0,M]$的一个划分,记为$T$,则$$m^-(D,T)=\sum_{i=1}^nm_i(x_i-x_{i-1}),$$
                    $$m^+(D,T)=\sum_{i=1}^nM_i(x_i-x_{i-1}),$$
                    所以$$\sum_{i=1}^nm_i(x_i-x_{i-1})\le m^-(D)\le m^+(D)\le \sum_{i=1}^nM_i(x_i-x_{i-1}),$$
                    又因为$f$可积,所以$\lim_{\lambda\to0}\sum_{i=1}^nm_i\Delta x_i=\lim_{\lambda\to0}\sum_{i=1}^nM_i\Delta x_i=\int_a^bf(x)\d x$,也即$m^-(D)=m^+(D)=\int_a^bf(x)\d x$,从而$D$是可求面积的,而且$m(D)=\int_a^bf(x)\d x$.
                \end{proof}
            \end{enumerate}
        \end{enumerate}
        \item 如何判断一个有界集$D$是可求面积的?\textbf{定义:}设$D\subset\mathbb{R}^2$是一个有界集,$D\subset U$,$U$是一个矩形,$T$是$U$的一个划分.$m^+(\partial D,T):=\sum_{U_{ij}\cap\partial D\not=\phi}|U_{ij}|=m^+(D,T)-m^-(D,T),m^+(\partial D)=\inf_Tm^+(\partial D,T)$.若$m^+(\partial D)=0$,则称$D$的边界面积是0,$D$是零边界集.
        
        \colorbox{pink}{例}
        \begin{enumerate}
            \item 当$f(x)\ge 0$且在$[a,b]$上可积,令$A=\{(x,f(x)):x\in[a,b]\}$,则$m^+(A):=\inf_Tm^+(A,T)=0$;
            \item 设$\Gamma=\{(\varphi(t),\psi(t)):t\in[a,b]\}$,其中$\varphi(t)$连续,$\psi(t)$连续可微,则$m^+(\Gamma)=0$.特别地,当$\Gamma$是光滑曲线时,$m^+(\Gamma)=0$;
            \item 如果只假设$\varphi(t),\psi(t)$连续,是否有$m^+(\Gamma)=0$?答案是否定的,如:Peano曲线.
        \end{enumerate}
    \end{enumerate}

    \paragraph{\colorbox{lime}{定理}}有界集$D$是可求面积的$\Leftrightarrow\partial D$面积是0.
    \begin{proof}
        ``$\Rightarrow$''若$D$可求面积,则$\forall\varepsilon>0,\exists U$的一个划分$T,s.t.$ $$m^+(D,T)<m+\varepsilon,m^-(D,T)>m-\varepsilon,m=m(D),$$所以
        $$m^+(\partial D,T)=m^+(D,T)-m^-(D,T)<2\varepsilon,$$从而$m^+(\partial D)<2\varepsilon$,所以$m^+(\partial D)=0$.\\
        ``$\Leftarrow$''若$m^+(\partial D)=0$,则$\forall\varepsilon>0,\exists U$的一个划分$T,s.t.m^+(\partial D,T)=m^+(D,T)-m^-(D,T)<\varepsilon,$所以$m^+(D)\le m^+(D,T),\forall T,m^-(D)\ge m^-(D,T)$,所以$0\le m^+(D)-m^-(D)<\varepsilon,$所以$m^+(D)=m^-(D)$,也即$D$可求面积.
    \end{proof}
    \fbox{Ex:}若$D_1,D_2$是两个可求面积的有界集,则$D_1\cup D_2,D_1\cap D_2,D_1\backslash D_2$都是可求面积的.(推论:若$\{D_i\}_{i=1}^m$是可求面积的,则$\cup_{i=1}^mD_i,\cap_{i=1}^mD_i$也是可求面积的.).

        \fbox{Ex:}若$D_1,D_2$是可求面积的,而且$D_1^o\cap D_2^o=\phi$,则$m(D_1\cup D_2)=m(D_1)+m(D_2)$(面积可加性).可加性同样可以推广到有限个集合.

    \subsection{2重Riemann积分:}
    \paragraph{\colorbox{lime}{定义}}(划分)$D\subset\mathbb{R}^2$是有界闭区域,零边界$T=\{\Delta D_1,\cdots,\Delta D_m\}$称$T$为$D$的一个划分,若
    \begin{enumerate}[1)]
        \item $\Delta D_i$都是可求面积的闭区域;
        \item (内部不交)$(\Delta D_i)^o\cap(\Delta D_j)=\phi,i\not=j$;
        \item $D=\cup_{i=1}^m\Delta D_i$.
    \end{enumerate}
    \paragraph{二重积分}设$D\subset\mathbb{R}^2$是一个零边界的有界闭区域,$f(x,y)$是$D$上的一个有界函数,$I\in \mathbb{R}$.如果$\forall\varepsilon>0,\exists\delta=\delta(\varepsilon)>0,s.t.$对$D$的任意划分$T=\{\Delta D_1,\cdots,\Delta D_m\},\forall(\xi_i,\eta_i)\in\Delta D_i$,只要$\lambda(T)<\delta,$都有$|\sum_{i=1}^Mf(\xi_i,\eta_i)\Delta\sigma_i-I|<\varepsilon$(其中$\lambda(T)=\max_{1\le i\le m}\{diam(\Delta D_i)\}$),则称$f(x,y)$在$D$上可积,$I$称为积分值,记为$\iint_Df(x,y)\d\sigma$或$\iint_Df(x,y)\d x\d y$.\\
    \textcolor{red}{Q:}如何来判断一个函数$f$是可积的?

    可积性理论$\begin{cases}
        Riemann-Darboux\text{理论:可积的充要条件}\\
        \text{可积函数类}
    \end{cases}$.

    Goal:去掉积分定义中的两个任意:``任意划分'',``任意取点''.设$D\subset\mathbb{R}^2$是一个零边界有界闭区域,$f(x,y)$是$D$上的有界函数,$T=\{\Delta D_1,\cdots,\Delta D_m\}$是$D$的一个划分,令$S(T)=\sum_{i=1}^mM_i\Delta\sigma_i,s(T)=\sum_{i=1}^mm_i\Delta\sigma_i$,其中$M_i=\sup_{\Delta D_i}f(x,y),m_i=\inf_{\Delta D_i}f(x,y)$.\textcolor{red}{Remark:}
    \begin{enumerate}
        \item $\forall(\xi_i,\eta_i)\in\Delta D_i,$有$s(T)\le\sum_{i=1}^mf(\xi_i,\eta_i)\Delta\sigma_i\le S(T)$;
        \item $\forall\varepsilon>0,\exists(x_i^1,y_i^1)\in\Delta D_i,(x_i^2,y_i^2)\in\Delta D_i$,s.t.$|f(x_i^1,y_i^1)-M_i|<\frac{\varepsilon}{m(D)},|f(x_i^2,y_i^2)-m_i|<\frac{\varepsilon}{m(D)}$,所以$|\sum_{i=1}^mf(\xi_i^1,\eta_i^1)\Delta\sigma_i-S(T)|<\varepsilon,|\sum_{i=1}^mf(\xi_i^2,\eta_i^2)\Delta\sigma_i-s(T)|<\varepsilon$.
    \end{enumerate}
    \paragraph{$S(T)$与$s(T)$的性质:}
    \paragraph{\colorbox{lime}{定义}}若$T=\{\Delta D_1,\cdots,\Delta D_m\},T'=\{\Delta D'_1,\cdots,\Delta D'_n\}$是$D$的两个划分,如果$\forall\Delta D'_i\in T',\exists\Delta D_j\in T,s.t.\Delta D'_i\subset\Delta D_j$,则称$T'$是$T$的一个加细.

    \begin{enumerate}[$\cdot$)]
        \item 由两个划分构造一个加细划分的方法:令$T\cup T':=\{\Delta D_i\cap\Delta D'_j\},\Delta D_i\in T,\Delta D'_j\in T'$且$\Delta D_i\cap\Delta D'_j\not=\phi$,则称$T\cup T$是$T$和$T'$的加细.
    \end{enumerate}

    \begin{enumerate}[1)]
        \item 划分加细,大和不增,小和不减.即若$T'$是$T$的一个加细,则$S(T')\le S(T),s(T')\ge s(T)$(证明与前面差不多,考虑一步)(约等于单调性));
        \item 任意两个划分$T,T'$,都有$s(T)\le S(T')$;
        \item $s(T)\le Mm(D),S(T)\le Mm(D)$,其中$M=\sup_D|f(x,y)|$.令$I_*=\sup_T\{s(T)\},I^*=\inf_T\{s(T)\}$,则$\forall$划分$T$,有$s(T)\le I_*\le I^*\le S(T)$.
    \end{enumerate}

    \paragraph{\textcolor{blue}{判别法1}}
    \paragraph{\colorbox{lime}{定理}}$f(x,y)$在$D$上可积$\Leftrightarrow \lim_{\lambda(T)\to0}(S(T)-s(T))=\lim_{\lambda(T)\to0}\sum_{i=1}^m\omega_i\Delta\sigma_i=0(\Omega_i=M_i-m_i)$.即$\forall\varepsilon>0,\exists\delta=\delta(\varepsilon)>0,s.t.\forall$划分$T$,只要$\lambda(T)<\delta$,则$S(T)-s(T)=\sum_{i=1}^m\omega_i\Delta\sigma_i<\varepsilon$.
    \begin{proof}
        $\Rightarrow:$若$f$在$D$上可积,则$\exists I\in\mathbb{R}$,s.t.$\forall\varepsilon>0,\exists\delta=\delta(\varepsilon)>0,s.t.\forall$划分$T,\forall(x_i,y_i)\in\Delta D_i,$只要$\lambda(T)<\delta$,都有$|\sum_{i=1}^mf(x_i,y_i)\Delta\sigma_i-I|<\frac{\varepsilon}{2}$,所以$|S(T)-I|<\varepsilon$(因为$\exists(x_i^1,y_i^1)\in\Delta D_i,s.t.|\sum_{i=1}^mf(x_i,y_i)\Delta\sigma_i-S(T)|<\frac{\varepsilon}{2}$),同理也有$|s(T)-I|<\varepsilon$.所以$|S(T)-s(T)|<2\varepsilon$,从而$\lim_{\lambda(T)\to0}|S(T)-s(T)|=0$.

        ``$\Leftarrow$''由于$s(T)\le I_*\le I^*\le S(T)$,所以$\forall\varepsilon>0,\exists\delta=\delta(\varepsilon)>0,s.t.\forall$划分$T$,只要$\lambda(T)<\delta$,则$S(T)-s(T)<\varepsilon$,所以$I^*-I_*\le S(T)-s(T)<\varepsilon$,从而$I^*=I_*=I$.又因为$S(T)-I\le S(T)-s(T)$,所以$\lim_{\lambda(T)\to0}S(T)=I$.同理$\lim_{\lambda(T)\to0}s(T)=I$,所以$\forall\varepsilon>0,\exists\delta=\delta(\varepsilon)>0,$当$\lambda(T)<\delta$时,$I-\varepsilon\le s(T)\le S(T)\le I+\varepsilon$,又$\forall(x_i,y_i)\in\Delta D_i$,有$s(T)\le \sum_{i=1}^mf(x_i,y_i)\Delta\sigma_i\le S(T)$,所以$I-\varepsilon\le \sum_{i=1}^mf(x_i,y_i)\Delta\sigma_i\le I+\varepsilon$,从而$f(x,y)$在$D$上可积.
    \end{proof}

    \paragraph{应用:}若$f(x,y)$在$D$上连续,则$f(x,y)$在$D$上可积.
    \begin{proof}
        $\forall\varepsilon>0,\exists\delta=\delta(\varepsilon)>0,s.t.\forall(x_1,y_1),(x_2,y_2)\in D,$只要$\sqrt{(x_1-x_2)^2+(y_1-y_2)^2}<\delta$,就有$|f(x_1,y_1)-f(x_2,y_2)|<\varepsilon$.对$D$的任意划分$T$,当$\lambda(T)<\delta$时,有$\omega_i=\sup_{\Delta D_i}f(x,y)-\inf_{\Delta D_i}f(x,y)<\varepsilon$(因$f$在$\Delta D_i$上的最大值与最小值可以取到).所以$\sum_{i=1}^m\omega_i\Delta\sigma_i<\varepsilon\sum_{i=1}^m\Delta\sigma_i<\varepsilon m(D)$,从而$f(x,y)$在$D$上可积.
    \end{proof}

    \paragraph{\textcolor{red}{Goal}}$f(x,y)$在$D$上可积$\Leftrightarrow\forall\varepsilon>0,\exists$划分$T=T(\varepsilon),s.t.\sum_{i=1}^m\omega_i\Delta\sigma_i<\varepsilon$.
    \paragraph{\colorbox{lime}{定理}}Darboux定理(可积性理论的基本定理) $\lim_{\lambda(T)\to0}S(T)=I^*=\inf_TS(T),\lim_{\lambda(T)\to0}s(T)=I_*=\sup_TS(T)$.
    \begin{proof}
        \begin{enumerate}[{Step }1]
            \item 由于$I^*=\inf_TS(T),$所以$\forall\varepsilon>0,\exists$划分$T_0=T_0(\varepsilon),s.t.0\le S(T_0)-I^*<\varepsilon$.
            \item 找$\delta=\delta(T_0)=\delta(\varepsilon)<<1,s.t.$只要划分$T$满足$\lambda(T)<\delta$,则$T$与$T\cup T_0$几乎相同,从而有$S(T)-S(T\cup T_0)<\varepsilon$;
            \item 只要$\lambda(T)<\delta$,有$S(T)-I^*=\underbrace{S(T)-S(T\cup T_0)}_{\le\varepsilon}+\underbrace{S(T\cup T_0)-S(T_0)}_{\le 0}+\underbrace{S(T_0)-I^*}_{\le\varepsilon}<2\varepsilon$.
        \end{enumerate}
    \end{proof}

    \paragraph{\textcolor{blue}{判别法2}}
    \paragraph{\colorbox{lime}{定理}}$f$在$D$上可积$\Leftrightarrow\forall\varepsilon>0,\exists$划分$T=T(\varepsilon),s.t.\sum_{i=1}^m\omega_i\Delta\sigma_i<\varepsilon$.
    \begin{proof}
        ``$\Rightarrow$'':easy.

        ``$\Leftarrow$'':只需证明$\lim_{\lambda(T)\to0}S(T)=I=\lim_{\lambda(T)\to0}s(T)$.由于$s(T)\le I_*\le I^*\le S(T),\forall T$,所以$I^*-I_*\le S(T)-s(T)=\sum_{i=1}^m\omega_i\Delta\sigma_i,\forall T$.由已知条件得,$\forall\varepsilon>0,$有$I^*-I_*<\varepsilon$,从而$I^*=I_*$.又有Darboux定理知,$\lim_{\lambda(T)\to0}S(T)=I=\lim_{\lambda(T)\to0}s(T)$.
    \end{proof}

    \paragraph{应用}记$A=\{(x,y)\in D|f\text{在}(x,y)\text{处取值不连续}\}$.若$m^+(A)=0$($A$的面积为0),则$f$在$D$上可积.特别地,若$f(x,y)$只在有限条光滑曲线段处不连续,则$f$可积.
    \begin{proof}
        由于$m^+(A)=0,$故$\forall\varepsilon>0,$存在$k=k(\varepsilon)$个长方形$\{H_i\}_{i=1}^m,s.t.A\subset\left(\cup_{i=1}^mH_i\right)^o,H^o_i\cap H^o_j=\phi,\sum_{i=1}^km(H_i)<\varepsilon$.记$Q=\cup_{i=1}^kH_i,$则$f$在$D\backslash Q^o$上连续,从而一致连续,所以对上述的$\varepsilon>0,\exists$划分$T'$s.t.$\sum_{T'}\omega_i\Delta\sigma_i<\varepsilon$.于是,$T'$与$H_i\cap D,\cdots,H_k\cap D$构成了$D$的一个划分,记为$\tilde{T}$.又$\sum_{\tilde{T}}\omega_i\Delta\sigma_i=\sum_{T'}\omega_i\Delta\sigma_i+\sum_{i=1}^k\omega_i\Delta\sigma_i\le\varepsilon+2M\varepsilon=(2M+1)\varepsilon$.
    \end{proof}

    \fbox{Ex1:}2元Riemann函数:
    $$R(x,y)=\begin{cases}
        \frac{1}{P_x}+\frac{1}{P_y},&x=\frac{q_x}{p_x},y=\frac{q_y}{p_y}\\
        0,&otherwise
    \end{cases},x,y\in[0,1],$$
    证明:$R(x,y)$在$[0,1]\times[0,1]$上可积,而且$\iint_{[0,1]\times[0,1]}R(x,y)\d x\d y=0$.

    \fbox{Ex2:}设$f$是$[a,b]\times[c,d]$上的一个函数,而且$\forall x\in[a,b],f(x,y)$关于$y$单调增加;$\forall y\in[c,d],f(x,y)$关于$x$单调增加,则$f(x,y)$在$D$上可积.

    \fbox{Ex3:}(定理)$f$可积$\Leftrightarrow\forall\varepsilon>0,\exists\sigma>0,\exists$划分$T=T(\varepsilon,\sigma)s.t.\sum_{\omega_i\ge\sigma}\Delta\sigma_i<\varepsilon$.

    \fbox{Ex4:}(复合函数的可积性)设$f$在$D$上可积且

    \paragraph{\textcolor{blue}{判别法2}}
    $f(x,y)$在$D$上可积$\Leftrightarrow\forall\varepsilon>0,\forall\sigma>0,\exists$划分$T=T(\varepsilon,\sigma),$s.t.$\sum_{\omega\ge\sigma}\Delta\sigma_i<\varepsilon$.
    \begin{proof}
        ``$\Rightarrow$'':$\forall\varepsilon>0,\forall\sigma>0,$存在划分$T=T(\varepsilon\sigma),$s.t.$\sum_T\omega_i\Delta\sigma_i<\varepsilon\sigma$;又$\sum_T\omega_i\Delta\sigma_i=\sum_{\omega_i\ge\sigma}\omega_i\Delta\sigma_i+\sum_{\omega_i<\sigma}\omega_i\Delta\sigma_i$,所以有$\sigma\sum_{\omega_i\ge\sigma}\Delta\sigma_i\le \sum_{\omega_i\ge\sigma}\omega_i\Delta\sigma_i<\varepsilon\sigma\Rightarrow\sum_{\omega\ge\sigma}\Delta\sigma_i<\varepsilon$.\\
        ``$\Leftarrow$'':$\forall\varepsilon>0,$取$\varepsilon=\sigma$,则$\exists T=T(\varepsilon,\sigma)=T(\varepsilon)$,s.t.$\sum_{\omega\ge\sigma}\Delta\sigma_i<\varepsilon$.所以
        \begin{align*}
            \sum_T\omega_i\Delta\sigma_i&=\sum_{\omega_i\ge\sigma}\omega_i\Delta\sigma_i+\sum_{\omega_i<\sigma}\omega_i\Delta\sigma_i\\
            &\le 2M\sum_{\omega_i\ge\sigma}\Delta\sigma_i+\varepsilon\sum_{\omega_i<\sigma}\Delta\sigma_i\le 2M\varepsilon+\varepsilon m(D),
        \end{align*}
        于是$f$在$D$上可积(由判别法2)
    \end{proof}

    \paragraph{应用}复合函数的可积性:若$f$在$D$上可积且$|f|\le M$,而且$g\in C([-M,M])$,则$g\circ f$在$D$上可积.
    \begin{proof}
        由$g\in C([-M,M])$,知$\forall\varepsilon>0,\exists\delta=\delta(\varepsilon)>0,$s.t.$\forall y_1,y_2\in[-M,M]$,且$|y_1-y_2|<\delta$,都有$|g(y_1)-g(y_2)|<\varepsilon$.又由于$f$可积,故$\forall\varepsilon>0$,以及上述的$\delta,\exists$划分$T=T(\varepsilon,\sigma)=T(\varepsilon)$,s.t.$\sum_{\omega_i(f)\ge\delta}\Delta\sigma_i<\varepsilon$.又$\omega_i(f)<\delta\Rightarrow\omega_i(g\circ f)\le\varepsilon$,所以$\sum_T\omega(g\circ f)\Delta\sigma_i=\sum_{\omega_i(g\circ f)\le\varepsilon}\omega_i(g\circ f)\Delta\sigma_i+\sum_{\omega_i(g\circ f)>\varepsilon}\omega_i(g\circ f)\Delta\sigma_i\le\varepsilon m(D)+2K\sum_{\omega_i(f)\ge\delta}\omega_i(g\circ f)\Delta\sigma_i(K=\sup_D|g\circ f|)\le\varepsilon m(D)+2K\varepsilon$,所有$g\circ f$在$D$上可积.
    \end{proof}
    \textcolor{blue}{复合函数可积性的另一个看法:}若$f(x_0,y_0)$是$f$的连续点,则$g\circ f$在$(x_0,y_0)$处连续,所以$\{(x_0,y_0):g\circ f\text{在}(x_0,y_0)\text{处不连续}\}\subset\{(x_0,y_0):f\text{在}(x_0,y_0)\text{处不连续}\}$,由Lebesgue定理知,$g\circ f$可积(\textcolor{blue}{Lebesgue定理:}$f$可积$\Leftrightarrow f$的不连续点是Lebesgue零测集).


    \paragraph{$n(n\ge 3)$重积分}
    \begin{enumerate}
        \item $\mathbb{R}^n$中有界集$\Omega$的体积:$\begin{cases}
            \text{如何定义?(内体积}V^-(\Omega),\text{外体积}V^+(\Omega),\text{可求体积与体积)}\\
            \text{如何判断}\Omega\text{可求体积?}\Omega\text{可求体积$\Leftrightarrow\partial D$的体积为0}\\
            %\text{\textcolor{red}{eg:}\textcolor{blue}{1)}}D\subset\mathbb{R}^n\text{是有界闭区域,}f\in C(D),\text{则}A=\{(\x,y):\x\text\in D,y=f(\x)\}\text{是}\mathbb{R}^{n+1}\text{中的零体积集};\text{\textcolor{blue}{2)}光滑曲面的体积是0,即:}A=\{\x\in\mathbb{R}^{n+1}:x_i=\varphi_i(\boldsymbol{t}),\boldsymbol{t}\in D\},\text{其中}D\subset\mathbb{R}^n\text{是一个有界闭区域,}\varphi_i(\boldsymbol{t})\in C^1(D)\text{而且Jacobi矩阵在$D^0$中满秩,}(C^1(D)=\{f\in C^1(D^0)|\partial_if\text{在}D^o\text{上一致连续}\}),\text{则}V^+(A)=0.\\
            \text{特别地,由有限片光滑曲面围成的区域是可求体积的}
        \end{cases}$
        \item $n$重积分$\begin{cases}
            \Omega\text{的划分}T=\{\Delta\Omega_1,\cdots,\Delta\Omega_m\}\\
            n\text{重积分(``分割''$\rightarrow$``求和''$\rightarrow$``求极限'')}
        \end{cases}$
        \item 可积性理论$\begin{cases}
            \text{Darboux大和与Darboux小和(单调性是关键)}\\
            \text{Darboux定理}\\
            \text{可积函数类}\begin{cases}
                \text{连续函数可积}\\
                \text{间断点的体积为0的函数可积}
            \end{cases}
        \end{cases}$
    \end{enumerate}


    \subsection{扩大可积函数类的方法}
    \begin{enumerate}
        \item 线性组合:若$f,g$在$\Omega$上可积,则$\forall\alpha,\beta\in\mathbb{R},\alpha f+\beta g$在$\Omega$上也可积(可推广到有限个函数的线性组合).
        \begin{proof}
            \begin{enumerate}[{方法}1]
                \item 用定义;
                \item $\omega_i(\alpha f+\beta g)\le |\alpha|\omega_i(f)+|\beta|\omega_i(g)$($\varepsilon-$ trick)
            \end{enumerate}
        \end{proof}
        \item 乘积可积性:若$f,g$在$\Omega$上可积,则$fg$在$\Omega$上也可积(可推广到有限个函数的乘积.特别地,$\forall k\in\mathbb{N}^+,f^k(x)$也可积).
        \begin{proof}
            $\omega_i(fg)\le \sup_{\Omega}|f|\omega_i(g)+\sup_{|D|}|g|\omega_i(f)$.
        \end{proof}
        \item 绝对可积性:若$f$可积,则$|f|$也可积.
        \begin{proof}
            $\omega_i(|f|)\le\omega_i(f)$.
        \end{proof}
        \item 复合可积性;
        \item 限制与延拓:若$f$在$\Omega$上可积,$\Omega_i\subset\Omega$是一个闭区域,零边界,则$f$在$\Omega_1$上可积.若$\Omega=\Omega_1\cup\Omega_2$且$\Omega_1^0\cap\Omega_2^0=\phi,f$在$\Omega_1,\Omega_2$上都可积,则$f$在$\Omega$上也可积.
        \begin{proof}
            $f$在$\Omega$上可积,所以$\forall\varepsilon>0,\exists\Omega$的划分$T=T(\varepsilon)=\{\Delta D_1,\cdots,\Delta D_m\}$,s.t.$\sum_T\omega_i(f)\delta\sigma_i<\varepsilon$.令$T^1=\{\Omega_1\cap\Delta D_i:i=1,2,\cdots,m\}$,则$T^1$是$\Omega_i$的一个划分,而且$\sum_{T^1}\omega_i(f)\Delta\sigma_i<\varepsilon$,从而$f$在$\Omega_1$上可积.
        \end{proof}
        \item 若$f$在$\Omega$上可积,$g$在$\Omega$上有界,而且$A=\{\x\in\Omega:f(\x)\not=g(\x)\}$的体积为0,则$g$在$\Omega$上也可积.
        \begin{proof}
            令$h(\x)=f(\x)-g(\x)$,只需证明$h(\x)$在$\Omega$上可积,而$h(\x)=0,\forall\x\in A^C$.由于$h(\x)$在$\Omega\backslash(\partial\Omega\cup\bar{A})$上是连续函数,而且$V^+(\partial\Omega)+V^+(\bar{A})=0$,从而$h(\x)$可积.
        \end{proof}
    \end{enumerate}
    




    \appendix
    \section{作业}
    \subsection{无条件极值}
    \begin{enumerate}[3.]
        \item 在$n$元$(n\ge 2)$函数中,存在有无穷多个极大值点,但无极小值点的函数.($n$元函数有鞍点)(1元连续函数:极小值点与极大值点是交替出现的)如函数$y=(1+\e^x)\cos x-y\e^y$.
    \end{enumerate}
    \begin{enumerate}[8.]
        \item \textcolor{blue}{方法一:}用隐式微分法寻找使得$y'(x_0)=0$的$x_0$;\\
        \textcolor{blue}{方法二:}即找函数$f(x,y)=y$在约束条件$g(x,y)=x^2+2xy+2y^2-1=0$的条件极值.取$L(x,y,\lambda)=y-\lambda(x^2+2xy+2y^2-1)$,令$$\begin{cases}
            \p{L}{x}(x,y,\lambda)=0\\
            \p{L}{y}(x,y,\lambda)=0\\
            g(x,y)=0
        \end{cases}$$
        由第二个式子知$\lambda\not=0$,代入第一个式子中,得$x+y=0$,再带入3式中,得$\begin{cases}
            x=1\\
            y=-1
        \end{cases}$或者$\begin{cases}
            x=-1\\
            y=1
        \end{cases}$,对应$\lambda_1=-\frac{1}{2},\lambda_2=\frac{1}{2}$.又$\frac{\partial^2L}{\partial x^2}=-2\lambda,\frac{\partial^2L}{\partial x\partial y}=-2\lambda,\frac{\partial^2L}{\partial y^2}=-4\lambda,$所以$(x,y)=(1,-1)$对应的$\mathrm{H}L=\begin{pmatrix}
            1&1\\
            1&2
        \end{pmatrix}>0,$所以$(1,-1)$是条件极小值点($\Rightarrow x=1$是$y(x)$的极小值点,并且极小值是$-1$);同理,$(-1,1)$是条件极小值点($\Rightarrow x=-1$是$y(x)$的极大值点,并且极大值是$1$).
        \fbox{自己算一下第9题}
    \end{enumerate}

    \begin{enumerate}[11.12.]
        \item $f$在区域$D$上的最值.关键是判断出$f(\x)$在$D$上存在最值.一般需演技据函数$f(\x)$在边界$\partial D$附近的行为,说明在$D$内一定可以取到最值.\textcolor{pink}{例:}$f(\x)$在$D$上$>0$,而且$\lim_{\substack{\x\in D \\ \x\to\partial D}}f(\x)=0$,则$f(\x)$在$D$内可以取到最大值.如12题中需要说明最大值不在边界上.
    \end{enumerate}

    \begin{enumerate}[13.]
        \item $f(x,y)=yx^y(1-x)$另一种方法:\begin{proof}
            $f(x,y)>0,\forall(x,y)\in D=\{(x,y):x\in(0,1),y\in(0,+\infty)\},\lim_{\substack{(x,y)\in D\\ (x,y)\to\partial D}}f(x,y)=0$,所以$f(x,y)$在$D$上可以取到最大值.令$\begin{cases}
                \p{f}{x}(x_0,y_0)=0\\
                \p{f}{y}(x_0,y_0)=0
            \end{cases},(x_0,y_0)\in D\Rightarrow\begin{cases}
                y_0(1-x_0)-x_0=0\\
                1+y_0\ln x_0=0(\Rightarrow x_0^{y_0}=\frac{1}{\e})
            \end{cases},$从而$f(x_0,y_0)=y_0x_0^{y_0}(1-x_0)=\frac{x_0}{\e}<\frac{1}{\e}$.
        \end{proof}
    \end{enumerate}


    \subsection{条件极值}
    \begin{enumerate}[1)]
        \item 条件极值的应用:证明不等式:例12.7.4,8,9,10题.做法:通过变形将不等式转化成条件极值问题,把未知数移到左边.以第9题为例:
        \begin{enumerate}[9.]
            \item $\forall a,b,c<0,$证明:$ab^2c^3\le 108\left(\frac{a+b+c}{6}\right)^6$.
            \begin{proof}
                \begin{align*}
                    ab^2c^3\le 108\left(\frac{a+b+c}{6}\right)^6&\Leftrightarrow\frac{ab^2c^3}{(a+b+c)^6}\le 108\left(\frac{1}{6}\right)^6\\
                    &\Leftrightarrow\frac{a}{a+b+c}\left(\frac{b}{a+b+c}\right)^2\left(\frac{c}{a+b+c}\right)^3\le 108\left(\frac{1}{6}\right)^6,
                \end{align*}
                从而我们只需考虑函数$f(x,y,z)=xy^2z^6$在约束条件$x+y+z=1,x,y,z>0$下的条件最大值.令$S=\{(x,y,z):x,y,z>0,x+y+z=1\}$,则$\lim_{\substack{(x,y,z)\in S\\ (x,y,z)\to\partial D}}f(x,y,z)=0$,而且$f(x,y,z)>0$ in $S$,从而$f$在$S$上可以取到最大值,于是可以用Lagrange乘数法计算.
            \end{proof}
        \end{enumerate}
        \item 齐次条件最值问题:设$f(\x)$是$k$次齐次函数,即$\forall\lambda>0,f(\lambda\x)=\lambda^kf(\x)$,约束条件:$g_i(\x)=\tilde{g}_i(\x)-a_i=0$,其中$\tilde{g}_i(\x)$是$k_i$次齐次函数.取$L(\x,\lambda_1,\cdots,\lambda_m)=f(\x)-\sum_{i=1}^m\lambda_ig_i(\x)$,令$\p{L}{x_j}(\x_0,\lambda_1^0,\cdots,\lambda_m^0)$,即$$\p{f}{x_j}(\x_0)-\sum_{j=1}^m\lambda_i^0\p{\tilde{g}_i}{x_j}(\x_0)=0(*)$$,且有$\tilde{g}_i(\x_0)=a_i$.我们的目标:计算$f(\x_0)$.在(*)两端乘$x_j$,并对$j$求和,得:
        \begin{align*}
            &kf(\x_0)-\sum_{i=1}^m\lambda_i^0k_i\tilde{g}_i(\x_0)\\
            =&kf(\x_0)-\sum_{i=1}^m\lambda_i^0k_ia_i=0
        \end{align*}
        (齐次函数的Euler公式),从而$f(\x_0)=\sum_{i=1}^m\frac{\lambda_i^0k_ia_i}{k}$,所以为计算条件最值,只需计算$\{\lambda_i^0\}_{i=1}^m$.
    \end{enumerate}
    





\end{document}